\documentclass[journal]{IEEEtran}

% *** FLOAT PACKAGES ***
%
%\usepackage{fixltx2e}
% fixltx2e, the successor to the earlier fix2col.sty, was written by
% Frank Mittelbach and David Carlisle. This package corrects a few problems
% in the LaTeX2e kernel, the most notable of which is that in current
% LaTeX2e releases, the ordering of single and double column floats is not
% guaranteed to be preserved. Thus, an unpatched LaTeX2e can allow a
% single column figure to be placed prior to an earlier double column
% figure.
% Be aware that LaTeX2e kernels dated 2015 and later have fixltx2e.sty's
% corrections already built into the system in which case a warning will
% be issued if an attempt is made to load fixltx2e.sty as it is no longer
% needed.
% The latest version and documentation can be found at:
% http://www.ctan.org/pkg/fixltx2e


%\usepackage{stfloats}
% stfloats.sty was written by Sigitas Tolusis. This package gives LaTeX2e
% the ability to do double column floats at the bottom of the page as well
% as the top. (e.g., "\begin{figure*}[!b]" is not normally possible in
% LaTeX2e). It also provides a command:
%\fnbelowfloat
% to enable the placement of footnotes below bottom floats (the standard
% LaTeX2e kernel puts them above bottom floats). This is an invasive package
% which rewrites many portions of the LaTeX2e float routines. It may not work
% with other packages that modify the LaTeX2e float routines. The latest
% version and documentation can be obtained at:
% http://www.ctan.org/pkg/stfloats
% Do not use the stfloats baselinefloat ability as the IEEE does not allow
% \baselineskip to stretch. Authors submitting work to the IEEE should note
% that the IEEE rarely uses double column equations and that authors should try
% to avoid such use. Do not be tempted to use the cuted.sty or midfloat.sty
% packages (also by Sigitas Tolusis) as the IEEE does not format its papers in
% such ways.
% Do not attempt to use stfloats with fixltx2e as they are incompatible.
% Instead, use Morten Hogholm'a dblfloatfix which combines the features
% of both fixltx2e and stfloats:
%
% \usepackage{dblfloatfix}
% The latest version can be found at:
% http://www.ctan.org/pkg/dblfloatfix

% Citation package
\usepackage{cite}

% graphic package
\ifCLASSINFOpdf
  \usepackage[pdftex]{graphicx}
  \DeclareGraphicsExtensions{.pdf,.jpeg,.png}
\else
  % \usepackage[dvips]{graphicx}
  % \DeclareGraphicsExtensions{.eps}
\fi

% math package
\usepackage{amsmath,amsfonts,amssymb,amscd,amsthm,xspace}
\usepackage{array}

% subfigure package
\ifCLASSOPTIONcompsoc
  \usepackage[caption=false,font=footnotesize,labelfont=sf,textfont=sf]{subfig}
\else
  \usepackage[caption=false,font=footnotesize]{subfig}
\fi

% tikz for graphs
\usepackage{tikz}
\usetikzlibrary{arrows,shapes,snakes,automata,backgrounds,petri, calc}

% other necesssary packages
\usepackage{multirow}
\usepackage{times}

\usepackage{breqn}
\usepackage{hyperref}

% correct bad hyphenation here
\hyphenation{spe-ci-fi-cal-ly re-gu-la-ted ge-ne-ra-li-sa-ti-on fun-ding dif-fe-rent pa-ra-me-ter a-ve-ra-ging ha-ving bia-ses va-lu-es in-fe-ren-ce theo-rem pro-ba-bi-li-ty ap-pro-xi-ma-ti-on ma-xi-mi-sing u-sing e-le-ments de-fi-ni-ti-on}

\DeclareMathOperator*{\argmax}{arg\,max}

\begin{document}

% some symbols for easy calls
\newcommand{\tp}{\textrm{\textit{tp}}}
\newcommand{\fp}{\textrm{\textit{fp}}}
\newcommand{\tpr}{\textrm{\textit{tpr}}}
\newcommand{\fnr}{\textrm{\textit{fnr}}}
\newcommand{\fpr}{\textrm{\textit{fpr}}}
\newcommand{\tnr}{\textrm{\textit{tnr}}}

% \title{Partially Observable Count Estimation by an Autonomous Mobile Robot}
\title{Partially Observable Poisson Processes for Mobile Robot Exploration}
% \title{Spectral-POPP for Robot Exploration in Partially Observable Environments}

\author{Ferdian~Jovan, Milan Tomy Mariya, Jeremy Wyatt, Nick Hawes 
% \IEEEcompsocitemizethanks{\IEEEcompsocthanksitem F. Jovan is with the Department
% of Engineering Science, University of Oxford, Oxford, OX1~3PJ.\protect\\
% E-mail: ferdian.jovan@gmail.com 
% \IEEEcompsocthanksitem Jeremy Wyatt is with the Department of Computer Science, University of Birmingham, Birmingham, B15~2TT.
% \IEEEcompsocthanksitem Nick Hawes is with Oxford Robotics Institute, Oxford, OX2~6NN. 
% }% <-this % stops an unwanted space
% \thanks{Manuscript received April 19, 2005; revised August 26, 2015.}
}

% The paper headers
\markboth{IEEE TRANSACTIONS ON ROBOTICS, VOL. XX, NO. X, MONTH YEAR}%
{Shell \MakeLowercase{\textit{et al.}}: Bare Demo of IEEEtran.cls for IEEE Journals}

% make the title area
\maketitle

\begin{abstract}
    The Poisson assumption is a popular choice when data arises in the form of counts. In many applications such as in mobile robotics, such counts are prone to a systematic error due to unreliable sensor algorithms and the data set that result can not be trusted. Only limited works have been done on the Poisson model when a full observability of the data is in question. We present practical Bayesian estimators in this paper for a partially observable Poisson process (POPP). Variations of these processes are presented, in which (i) sensors are uncorrelated, (ii) sensors are correlated, (iii) the unreliability of the observation model, when built from data, is accounted for. The actual posterior distributions are estimated via two tractable approximations, which we combined in a switching filter. The filter enables efficient and accurate estimation of the posterior. A detailed empirical analysis is performed on both simulated and real-world data, and an application of the resulting posterior to drive exploration of a mobile robot with unreliable sensors is presented. 
\end{abstract}

\begin{IEEEkeywords}
Poisson processes, partial observability, misclassified counts, robot exploration
\end{IEEEkeywords}


\IEEEpeerreviewmaketitle

\section{Introduction}
\label{sec:introduction}

% Autonomous mobile robots that work in human-centred environments are seen as a promising future application area for robotics systems. These systems are expected to leave carefully controlled industrial environments and come to assist us with menial tasks in our homes and offices. Possible applications include, for example, domestic robot assistants which help people to live independently for longer \cite{5751968}, delivery robots in hospitals, stock-keeping robots in supermarkets and warehouses, and security robots in factories.
% 
% Having robots operate in human populated environments requires modelling human activities. This is because any tasks or plans the robots have must take into account human activities. Since human activities involve many complex interactions they can be modelled at many levels of detail. These can range from recognising simple activities over a few seconds, a minute, or an hour; predicting what a person is going to do next; to determining whether a group of people are performing an activity together. In any scenario where a robot learns about human activities, it must first observe them. Thus, the first thing the robot needs to do is to plan to go to where people are. This problem of finding and observing people is the basic motivation for this thesis.  
% 
% To be where people are, the robot must first know when and where it is likely to see people. It becomes, however, a challenging problem if one tries to predict exactly where a particular individual will be, so as to observe that individual. There are two reasons for this. First, each individual roaming in an human populated environment is hard to re-recognise. Second, individual persons often travel long distances and visit places robots can not follow. Hence, instead of predicting where an individual will be, it would be easier to predict where the robot is likely to see many people, without regard for exactly who it may observe. This formulation would allow the robot to observe a sufficient amount of human activity, so as to learn the specific kinds of activity models mentioned previously.
% 
% The problem of this thesis can be loosely formulated as how to predict where many people are most likely to be and to go and observe them. Specifically, it requires the robot to go to where the aggregate level of human activity is highest. In addition, this thesis chooses to tackle the problem for the case where the robot runs for an extended period of time such days, or even weeks as it builds its models.
% 
% To tackle the formulated problem, a mobile robot must know not only where people are, but also when they are in those locations. It also needs a model which predicts how many people the robot will be likely to see in a particular place at a particular time, since people tend to move around from one location to another. 
% 
% An important restriction on using a mobile robot is that it can only be in one place at one time, so its observations are spatially restricted. As the robot moves around, it will only see particular locations infrequently. Thus its data for those locations will be temporally sparse. This adds yet another requirement to the problem formulation, where the robot must know how uncertain it is about how many people might be in one place at a particular time. This requirement is necessary since the robot has a limit to its operational life. We would therefore like it to optimise the time it takes to build its models. This introduces an exploration-exploitation trade-off problem \cite{wyatt1998exploration, 1413255, AUDIBERT20091876}, i.e. should the robot visit a familiar place, where it will probably see two people, or go to a new place, where it might see many more but might see no-one?  Another important restriction in mobile robotics is that robot sensors are unreliable. Any solution must take into account the unreliability of sensors. We may also require that it do so for when multiple sensors are involved.

\section{Preliminaries - the Spectral-FOPP}
\label{sec:preliminaries}

Our work is built on top of the work in \cite{jovan_iros16} which is able to extract temporal dynamics in the aggregate level of human activities to predict human activity level at particular times and places. Hence, this section is dedicated to briefly explain the technique employed in \cite{jovan_iros16}.

\subsection{the FOPP}

A fully observable Poisson process (FOPP) is a counting process $N(t_1, t_2)$ where a counter tells, with perfect accuracy, the number of events that occurred during a specified interval ($t_1,t_2$). $N(t_1,t_2) = x_i$ states that in the $i$-th observation of interval ($t_1, t_2)$, there are $x_i$ events. The number of events $N(t_1, t_2)$ in a finite interval of length $t = t_2 - t_1$ follows the Poisson distribution, 
\begin{equation}
    \label{eq:pmf_poisson}
	Poi(N(t_1, t_2) = x \mid \lambda) = \frac{e ^{-\lambda} \lambda ^{x}}{x!}
\end{equation}
\noindent where $\lambda$ represents the {\em arrival rate, mean count, intensity}, or {\em expected number of events} in a fixed interval $(t_1,~ t_2)$. Here we refer to $\mathcal{N}(t_1, t_2)$ as a measurement of the number of individuals or objects detected over the time interval $[t_1, t_2)$. $\lambda$ is thus transformed into a function of time, i.e. $\lambda(t_1, t_2)$. Hence, (\ref{eq:pmf_poisson}) becomes a non-homogeneous Poisson process, in which the degree of heterogeneity depends on the function $\lambda(t_1, t_2)$. As we use a fixed time interval at any point in time, we define $\lambda(t_i, t_{i+\delta})$ for $i \in \{1,\ldots,T\}$ and $\delta \in \mathbb{N}$.

Bayesian estimation for fully observable Poisson processes relies on the conjugacy between the Poisson and a Gamma density
\[
\begin{tabular}{rcl}
$Gam(\lambda \mid \alpha, \beta)$ & = & $\displaystyle\frac{\beta ^{\alpha}}{\Gamma (\alpha)} \lambda ^{\alpha - 1} e^{-\beta \lambda}$ \\ [1ex]
\end{tabular}
\]
%\noindent where $\Gamma (\alpha)$ is the Gamma function ($(\alpha -1)!$ for integers and $\int_{0}^{\infty} x^{\alpha-1} e^{-x} dx$ for non-integers), 
where $\alpha, \beta$ are the shape and the rate parameters. The posterior is thus also Gamma:
\begin{equation}
\label{eq:bayes_poisson}
    \begin{array}{lll}
    P(\lambda \mid x_1, \ldots, x_n) & \varpropto Poi(x_1, \ldots, x_n \mid \lambda) ~ Gam(\lambda \mid \alpha, \beta) \\
     & = Gam \Bigg(\lambda \mid \displaystyle\sum_{i=1}^{n} x_i + \alpha, n + \beta \Bigg)
\end{array}
\end{equation}

On each posterior update, we choose the \textit{Maximum a Posteriori (MAP hypothesis) $\lambda_{map}(t_i, t_j)$} to be the point estimate of $\lambda(t_i, t_j)$. A collection of MAP estimates ordered from $\lambda_{map}(t_0, t_0 + \delta)$ to $\lambda_{map}(t_{\delta-1}, t_{\delta-1} + \delta)$ creates a MAP time series. 

\subsection{Fourier Representation of the FOPP}

To capture the periodic structures over the $\lambda$ function, i.e. $\lambda(t_i, t_j)$ of a Poisson process, the Fourier transformation is proposed which offers a fast transformation and re-transformation. The periodic structures are believed governing aggregated human activities such as daily, weekly, or even hourly and exploiting these structures improve the prediction accuracy of where and when the aggregate human activities tend to happen \cite{jovan_iros16}.

The \textit{Fourier transform} is a reversible, linear transformation that decomposes a function of time $f(t)$ into the frequencies $F(\omega)$ that make it up. $F(\omega)$ is formed of \textit{complex exponentials}. A complex exponential is a complex number in the form of 
\begin{equation*}
    e^{i\theta} = cos(\theta) + i~sin(\theta)
\end{equation*}
which is a point on the unit circle at an angle of $\theta$. For any given complex exponential $e^{i\theta} = cos(x) + i~sin(x)$, the \textit{absolute value} and \textit{argument} which correspond to the amplitudes and phase shifts of the frequency components $\omega$ can be obtained.

As the parameter $\lambda$ of a Poisson process was defined as a function of time, i.e. $\lambda(t_1, t_2)$, the periodic patterns of parameter $\lambda(t_1, t_2)$ can be extracted using the Fourier transform by calculating the frequency spectrum $F(\omega)$ of $\lambda(t_i, t_j)$, i.e. $F(\omega) = \mathcal F(\lambda(t_i, t_j))$. Once $\lambda(t_i, t_j)$ is in frequency domain, a spectral analysis on $F(\omega)$ can be carried out. One simple spectral analysis is to select $l$ frequency components $\omega_k$ (for $k=1,\ldots,l$) with the highest absolute value creating a new frequency spectrum $F'(\omega)$. Then the inverse Fourier transformation is performed on $F'(\omega)$ to reconstruct a smooth periodic function of $\lambda(t_i, t_j)$, i.e. $\lambda'(t_i, t_j) = \mathcal F^{-1}(F'(\omega))$.

\section{Related Work}
\label{sec:related}

There are several variations of the basic Poisson process which has been recently used to model regularities in time series data in order to identify the irregular ones. Ihler et al. \cite{Ihler2006adaptive, Ihler2007detecting} described a modified Markov-modulated Poisson processes for detecting unusual data points or segments in time-series. The Poisson processes are used as probabilistic models for counting regular patterns and behaviour whereas the Markov chain is used to track the occurrence of anomalous events. Hutchins in \cite{hutchins2007countdata} extended the work of \cite{Ihler2006adaptive} from single to multiple counters, and applied it to estimating the occupancy level of a building. 

The work of \cite{hutchins2007countdata} also took into consideration the effect of misclassification (under-or-over count) on the Poisson distribution. Only a few studies are found studying this effect. Bratcher and Stamey used a Bayesian method to estimate Poisson rates in the presence of both undercounts and overcounts. They used the double sampling technique where the first sample is searched with both a fallible and an infallible method and the second sample is searched with only a fallible method \cite{bratcher2002}. It was then extended to a fully Bayesian method for interval prediction of the unobservable actual count in a future sample, given a current double sample \cite{bratcher2004}. Stamey and Young \cite{stamey2005} managed to obtained closed-form expressions for maximum likelihood estimators (MLEs) of the false negative rate, the false positive rate, and the Poisson rate for the model proposed in \cite{bratcher2002}. The estimators are straightforward to calculate and to interpret in terms of evaluating the effectiveness of using unreliable counts.

% This work is closely related to  \cite{hutchins2007countdata}. They used multiple unreliable counters, each at a different exit or entrance. Thus, each sensor is associated with a different Poisson process. For each entrance or exit they used a MMNHPP to estimate the arrival rate and a noise model to capture under-and over-counting. Their interest is in estimating a single latent variable influencing the arrival rates at multiple exits or entrances. In our case, we are interested in multiple unreliable sensors estimating the parameter of a single Poisson process.  

Our work is an extension to the work in \cite{jovan18a} in accurately estimating the parameter of a single Poisson process to improve the prediction accuracy of Spectral-FOPP. We present three further extensions of the POPP model and its application to exploit the temporal dynamics in the aggregate level of human activities for better robot exploration.

\section{The POP Process}
\label{sec:popp}

The partially observable Poisson process (POPP) is a counting process $N(t)$ with arrival rate $\lambda$ where the number of events that occurred up to time $t$ are obtained by unreliable (possibly multiple) counters. The definition brings a distinction between \emph{true count} (or simply \emph{count}), which refers to the number of events that actually happened, and the \emph{sensed count}, which refers to the count obtained by a counter (or sensor). Given that $c_i$ number of events happened (as the true count) over the interval $[0, t)$ during the $i$-th observation, and $m$ counters observed the events unreliably, thus the sensed count $s_{ji}$ is the count given by sensor $j$ in the $i$-th observation within the interval $[0, t)$ with $0 \leq j \leq m$. 

\begin{figure}[t!]
	\centering
	\includegraphics[width=0.5\textwidth]{./figures/gm_popp.jpg}
    \caption{Graphical representation of the partially observable Poisson process.}
	\label{fig:gm_popp}
\end{figure}

The graphical model, which is easily derived from the definition of the POPP, shows that the true count $c_i$ has become a latent variable which can only be inferred from the sensed count $\overrightarrow{s_i} = (s_{1i}, \ldots, s_{mi})$ where each $s_{ji}$ is a sensed count coming from sensor $j$. The posterior of $\lambda$ is then inferred from the posterior of $c_i$ after multiple samples $i = 1 \ldots n$.

A statistical inference to estimate the rate parameter $\lambda$ of the POPP model is done with a marginalisation over all possible true count value $c_i$ in a joint distribution between the posterior $P(\lambda ; c_i)$ and the posterior over $c_i$ given $\overrightarrow{s_i}$. The posterior of $\lambda$, given $n$ samples $\overrightarrow{s}=(\overrightarrow{s_1} \dots \overrightarrow{s_n})$, each consisting of $m$ sensors, is:
\begin{equation}
	\label{eq:marginal_occurrences}
	\begin{tabular}{r@{=}l}
		$P(\lambda ; \overrightarrow{s})$ &  $\displaystyle\sum_{c_1=0}^{\infty} \ldots \displaystyle\sum_{c_n=0}^{\infty} P(\lambda ; \overrightarrow{c}) ~ P(\overrightarrow{c} ; \overrightarrow{s})$ \\
	\end{tabular}
\end{equation}
\noindent where
\begin{equation*}
	\begin{tabular}{r@{ = }l}
		$P(\lambda ; \overrightarrow{c})$ & $Gam\Bigg(\lambda ; \displaystyle\sum_{i=1}^{n} c_i + \alpha, n + \beta \Bigg)$
	\end{tabular}
\end{equation*}
\noindent with $\overrightarrow{c} = (c_1, \ldots, c_n)$ for $1 \leq i \leq n$.

$P(\overrightarrow{c} ; \overrightarrow{s})$ is factored based on the assumption that each sensor is \textit{uncorrelated} to one another given the true count $c_i$. Consequently, the probability that the vector of true counts is $\overrightarrow{c}$, given $n$ samples of the vector of $m$ sensed counts $\overrightarrow{s_1}, \ldots, \overrightarrow{s_n}$, is

\begin{equation}
    \label{eq:occurrences_likelihood}
    \begin{tabular}{r@{ $\varpropto$ }l}
        $P(\overrightarrow{c} ; \overrightarrow{s_1}, \ldots, \overrightarrow{s_n})$ & $P(\overrightarrow{s_1}, \ldots, \overrightarrow{s_n} ; \overrightarrow{c}) ~ P(\overrightarrow{c})$ \\ [1ex]
        & $\displaystyle\prod_{i=1}^{n} P(\overrightarrow{s_i} ; c_i) ~ P(c_i)$ \\ [2ex]
        & $\displaystyle\prod_{i=1}^{n} \displaystyle\prod_{j=1}^{m} P(s_{ji} ; c_i) ~ P(c_i ; \overrightarrow{c_{-1}})$
    \end{tabular}
\end{equation}

\noindent where $\overrightarrow{c_{-1}} = c_{i-1}, \ldots, c_1$.

$P(c_i ; \overrightarrow{c_{-1}})$ and $P(s_{ji} ; c_i)$ are defined to complete Eqn. \ref{eq:occurrences_likelihood}. $P(c_i ; \overrightarrow{c_{-1}})$ is calculated in the form of a negative binomial distribution

\begin{equation}
	\label{eq:unconditional_xi}
	\begin{tabular}{r@{=}l}
		$P(c_i ; \overrightarrow{c_{-1}})$ & $\displaystyle\int_{\lambda=0}^{\infty} P(c_i ; \lambda) ~ P(\lambda ; \overrightarrow{c_{-1}}) ~d\lambda$ \\ [2ex]
		& $\displaystyle\int_{\lambda=0}^{\infty} Poi(c_i ; \lambda) ~ Gam(\lambda ; \alpha, \beta) ~d\lambda$ \\ [2ex]
		& $NB\Bigg(c_i ; \alpha, \displaystyle\frac{\beta}{\beta + 1}\Bigg)$.
	\end{tabular}
\end{equation}

The Poisson limit theorem states that the Poisson distribution may be used as an approximation to the binomial distribution \cite{papoulis2002probability}. Using this theorem as the foundation, an arbitrarily close approximation to the probability $P(s_{ji} ; c_i)$ is defined by assuming there exists a small enough finite subinterval of length $\delta$ for which the probability of more than one event occurring is less than some small value $ \epsilon$. With this assumption, interval $[0, t)$ is splitted into $l$ smaller subintervals $I_1, \ldots, I_l$ of equal size, with the condition that $l > \lambda$ (the condition is crucial since we focus on very small portions of the interval). Consequently, the whole interval $[0, t) = I_1, \ldots, I_l$ becomes a series of Bernoulli trials, where the $k^{th}$ trial corresponds to whether (1) an event $e_k$ happens with probability $\lambda / l$ and (2) a sensor $j$ captures the event $e_k$ as the detection $d_k$ at the subinterval $I_k$.

$P(s_{ji} ; c_i)$ is defined as the aggregate of the true positives $tp_{ji}$ in $c_i$ subintervals, and the false positives $fp_{ji}$ in $l-c_i$ subintervals. The probability of a \textit{true positive detection} (TP) for sensor $j$ in a subinterval is $tpr_j = P_j(d = 1 ; e=1)$, and the probability of a \textit{false positive detection} (FP) is $fpr_j = P_j(d = 1 ; e=0)$. Thus $P(s_{ji} ; c_i)$ is defined as a sum of two binomial distributions $B(r ; n,\pi)$, where the aggregate is constrained to be $s_{ji}$: 

\begin{equation}
	\label{eq:joint_binomial_distribution}
    P(s_{ji} ; c_i) \! = \! \! \! \displaystyle\sum_{\textrm{tp}_{ji} = 0}^{c_{i}} \! \! B\Big(\textrm{tp}_{ji} ; c_i, \textrm{tpr}_j\Big) B\Big(\textrm{fp}_{ji} ; \Delta c_i, \textrm{fpr}_j \Big)
\end{equation}
\noindent where $s_{ji} = \textrm{tp}_{ji} + \textrm{fp}_{ji}$, $\textrm{tpr}_j = P_j(d=1 ; e=1)$, $\textrm{fpr}_j = P_j(d=1 ; e=0)$, and $\Delta c_i = (l - c_i)$.

Eqn.~\ref{eq:marginal_occurrences} shows the difficulty of belief state estimation in the POPP model since there is no conjugate density accomodating an analytical solution for the posterior over $\lambda$. Each sensed count sample $\overrightarrow{s_i}$ used to update the posterior of $\lambda$ adds a factor of countably infinite number of elements. The resulting posterior is a sum of countably infinite sums. One can place an upper bound $l$ on the maximum value of each $c_i$, but it still makes the number of elements in the posterior grow by a factor $l$ with every sensed count $\overrightarrow{s_i}$.  

With this difficulty noted, Jovan et al., proposed three efficient estimators, each of which offers an approximation to the true posterior $P(\lambda ; \overrightarrow{s})$. A more detailed presentation of these estimators is given in \cite{jovan18a}.

% \input{src/filter.tex}
%!TEX root = ../bare_jrnl.tex

\section{The POPP Extensions}
\label{sec:popp_extensions}

In~\cite{jovan18a} demonstrated that the POPP model is able to efficiently correct miscounts made by multiple unreliable counting devices observing a single Poisson process. However, the POPP model is limited by two assumptions:
\begin{enumerate}
    \item the sensors are conditionally independent given the true count, and 
    \item the degree of the unreliability of a sensor is precisely known.
\end{enumerate}
In this paper, we propose three extensions to the POPP model to tackle these assumptions. The first extension (C-POPP) modifies the POPP model to accommodate correlation between sensors. The second extension (POPP-Beta) extends the POPP model with an observation model which captures uncertainty over the sensor reliability. The third extension (POPP-Dirichlet) combines these ideas to jointly address both assumptions. 

\subsection{The Correlated POPP}
\label{subsec:cpop}

Recall that Eqn.~\ref{eq:occurrences_likelihood} is defined under the assumption that each sensor count is conditionally independent from all the others given the true count. This assumption ignores the correlations between sensors. 
% 
To introduce correlations between sensors we must alter Eqn.~\ref{eq:independent_sensor_likelihood} and
Eqn.~\ref{eq:joint_binomial_distribution} from the POPP model.

\textbf{NOTE: I think we are mixing observation and detection a bit (and even capture earlier). Also sensor model and observation model are used for the same thing in different places.}
For Eqn.~\ref{eq:joint_binomial_distribution}, recall that the probability of a particular sensed count was defined from the Poisson limit theorem as a sequence of Bernoulli trials over subintervals.
% 
With correlated sensors, the observation of an event $e_k$ in the 
$k^th$ trial no longer follows the Bernoulli distribution. Instead it follows the categorical distribution, where the $k^{th}$ trial corresponds to whether a particular combination of binary detections $d_{1,k}, \ldots, d_{m,k}$ happens in subinterval $I_k$. Therefore, instead of having an independent sensor models for the detection of event $e_k$ we proposed a joint observation model:
\begin{equation}
    \label{eq:joint_sensor_model}
    P_{jnt}(\vec{d_k} ; e_k)
\end{equation}    
\noindent where $ \vec{d_k} = (d_{1,k}, \ldots, d_{m,k})$, with $d_{j,k}$ being a detection by sensor $j$ in the $k^th$ subinterval (trial), and $d_{j,k}, e_k \in {0, 1}$. 

From this we define functions $\mathcal E^+, \mathcal E^- : \vec{d} \rightarrow [0,1]$ which provide the probability of a joint observation given that $e_k$ occurred or did not, respectively. 

\begin{equation}
\mathcal E^+ = P_{jnt}(\vec{d_k} ; e_k=1)
\end{equation}
\begin{equation}
\mathcal E^- = P_{jnt}(\vec{d_k} ; e_k=0)
\end{equation}


\textbf{NOTE: Probably not needed but this implies that the value of each function over all $\vec{d_k}$ sums to 1. \emph{$\mathcal E^+$ and $\mathcal E^-$, each sums up to 1. These $\mathcal E^+$ and $\mathcal E^-$ are basically the TPR and TNR in these joint observation/sensor models}}

Recall that the set of detections for observation period $i$ was defined as:
\begin{equation*}
    \vec{s_i} = (s_{1i}, \ldots, s_{mi})
\end{equation*}
with $s_{j,i} \in \mathbb N$ and $s_{j,i}$ the sensed count of sensor $j$ from the $i$-th observation period. Since the joint observation model is defined under a combination of binary detections of sensors, each $s_{j,i}$ can be split into $l$ subintervals such that in each sub interval $I_k$ there is only one detection $d_{j,k}$. If the binary detections from all sensors at subinterval $I_k$ are grouped together, then for the observation period $i$, $\vec{s_i}$ can be alternatively defined as a list of $l$ \emph{detection groups} of binary detections:
\begin{equation}
    \label{eq:s_i_definition}
    \vec{s'_i} = ((d_{1,1}, \ldots, d_{m,1}), \ldots, (d_{1,l}, \ldots, d_{m,l}))
\end{equation}
\noindent with $d_{j,k}$ being a detection by sensor $j$ at subinterval $I_k$ and $d_{j,k} \in \{0, 1\}$. Note that this is not a set since detection groups can be duplicated across subintervals.

In order to define the joint probability of a particular count being yielded by a particular sequence of detection groups, we must consider all possible combinations of true positives and false positives that could be generated by that sequence. We do this in the following definition of $P(\vec{s_i} ; c_i)$, and define the probability of a given sequence of detection groups yielding count $c_i$ using the multinomial distribution.

\begin{equation}
\label{eq:codependent_sensor_likelihood}
P(\vec{s_i} ; c_i) = \sum\limits_{s \in \mathcal{P}(\vec{s'_i})} Multi(s ; |s|, \mathcal E^+) ~ Multi(\vec{s'_i}\setminus s ; (c_i - |s|), \mathcal E^-)
\end{equation}

\noindent where $\mathcal{P}(\vec{s'_i})$ is the powerset of $s'_i$ and $|s'_i| = c_i$. \textbf{NOTE: \emph{Is using powerset correct? How do we distinguish the total number of each $(d_{1,1}, \ldots, d_{m,1}$?}}

Eqn.~\ref{eq:codependent_sensor_likelihood} can be understood by analogy to Eqn.~\ref{eq:joint_binomial_distribution}. In both equations all possible ways pairs of true and false positives counts which sum to $c_i$ are considered. In the conditionally independent case the binomial distribution is used to determine the probability of each count from the available trials given the true and false positive rates. In the correlated case the multinomial distribution is to determine the probability of each count from a possible sequence of joint observations and their probability of yielding an observation.

One should note that the benefit of C-POPP is that it exploits correlations among multiple sensors contributing to detection counts. If there is only one sensor counting events, then the POPP model is more computationally efficient.

%!TEX root = ../bare_jrnl.tex

\subsection{The POPP-Beta}
\label{subsec:popb}

The POPP model requires the true positive and false positive rates to the specified for sensor $j$, i.e.  $\tau_j = P_j(d;e=1)$ and $\xi_j = P_j(d;e=0)$. 
The POPP model requires these rates to be accurate in order to generate correct posteriors over $\lambda$. To accurately determine the rates in practice, one needs to have a large data set of both sensed counts and the ground truth. Given the ground truth is typically manually created, this places a lot of burden on experts who need to label the data.   

Here, we extend the original POPP model to take into account uncertainty in the true and false positive rates due to limited training data.
% 
To model this uncertainty we use Bayesian estimation to determine the 
true positive rate ($\tau$) and false positive rate ($\xi$)
% 
We use Beta distributions as priors for $\tau$ and $\xi$ because the beta distribution act as a conjugate to the binomial distribution, providing a family of prior probability distributions for the parameter of a binomial distribution. The beta-binomial conjugacy leads to an analytically tractable compound distribution called the beta-binomial distribution ($BB(d ; c, \zeta, \eta)$), where the $p$ parameter in the binomial distribution $B(d ; c, p)$ is drawn from a beta distribution $Be(p ; \zeta, \eta)$.

% \begin{equation}
% 	\label{eq:beta_binomial}
% 	\begin{tabular}{r@{ = }l}
% 		$P(d ; c, \zeta, \eta)$ & $\displaystyle\int_{0}^{1} P(d ; c, p) ~ P(p ; \zeta, \eta) ~dp$ \\ [2ex]
% 		& $\displaystyle\int_{0}^{1} B(d ; c, p) ~ Be(p ; \zeta, \eta) ~dp$ \\ [2ex]
%         & $\displaystyle\int_{0}^{1} \binom{c}{d} p^d (1-p)^{(c-d)} ~ \displaystyle\frac{p^{(\zeta - 1)} (1-p)^{(\eta-1)}}{\pi(\zeta, \eta)}$ \\ [2ex]
%         & $\displaystyle\binom cd \displaystyle\frac{1}{\pi(\zeta, \eta)} \displaystyle\int_{0}^{1} p^{(d+\zeta-1)} (1-p)^{(c-d+\eta-1)} dp$ \\ [2ex]
%         & $\displaystyle\binom cd \displaystyle\frac{\pi(d+\zeta, c-d+\eta)}{\pi(\zeta, \eta)}$ \\ [2ex]
%         & $BB(d ; c, \zeta, \eta)$
% 	\end{tabular}
% \end{equation}

Our sensor rates are now estimated from two beta distributions: $Be(\tau ; \zeta_{\tau}, \eta_{\tau})$ and $Be(\xi ; \zeta_{\xi}, \eta_{\xi})$.
% 
$\zeta_{\tau}$ and $\zeta_{\xi}$ are the number of true positive and false positive detections in the ground truth data respectively.
% 
$\eta_{\tau}$ and $\eta_{\xi}$ are the number of false negative and true  negative detections in the ground truth data respectively. 
% 
Given these parameters, we form the POPP-Beta model from POPP by replacing Eqn. \ref{eq:joint_binomial_distribution} with:  

\begin{equation}
	\label{eq:joint_beta_binomial_distribution}
    P(s_{j,i} ; c_i) \! = \! \! \! \displaystyle\sum_{r = 0}^{c_{i}} \! \! ~ BB\Big(r ; c_i, \zeta_{\tau}, \eta_{\tau}) \Big) BB\Big( (s_{j,i} - r) ; (c_i - r), \zeta_{\xi}, \eta_{\xi}) \Big)
\end{equation}


% With a sensor model which follows beta densities and is fully integrated, as a distribution, in the sensed count likelihood $P(s_{ji} ; c_i)$ as shown in Equation \ref{eq:joint_beta_binomial_distribution}, we obtain a graphical model with the structure shown in Figure \ref{fig:gm_popp_beta}.

One should note that the difference between the POPP and POPP-Beta model, lies only in the change from Eqn. \ref{eq:joint_binomial_distribution} to \ref{eq:joint_beta_binomial_distribution}. However, given little training data for the observation model, the POPP-Beta model is expected to be more conservative in estimating the posterior $P(\lambda ; \vec{s})$ over $\lambda$ than the POPP model. 

% \begin{figure*}[t!]
% 	\centering
% 	\begin{tikzpicture}
% 	\tikzstyle{place}=[rectangle,draw=blue,thick,minimum size=5 mm]
% 	\tikzstyle{empty}=[rectangle,draw=white,thick,minimum size=5 mm]
% 	\tikzstyle{every label}=[black]
% 	\begin{scope}
% 	\node[place](31)[xshift=0mm]{$S_{1i}$};
% 	\node[place](32)[right of=31, xshift=20mm]{$S_{2i}$};
% 	\node[place](33)[right of=32, xshift=20mm]{$\ldots$};
% 	\node[place](34)[right of=33, xshift=20mm]{$S_{(m-1)i}$};
% 	\node[place](35)[right of=34, xshift=22mm]{$S_{mi}$};
% 	\node[place](21)[above of=33]{$X_i$} edge[post](31) edge[post](32) edge[post](33) edge[post](34) edge[post](35);
% 	\node[place](11)[above of=21]{$\lambda$} edge[post](21);
%     \node[place](12)[right of=11, xshift=20mm]{$\textrm{tpr}_{(m-1)}, \textrm{xi}_{(m-1)}$} edge[post](34);
%     \node[place](13)[left of=11, xshift=-20mm]{$\textrm{tpr}_{2}, \textrm{xi}_{2}$} edge[post](32);
%     \node[place](14)[right of=12, xshift=22mm]{$\textrm{tpr}_{m}, \textrm{xi}_{m}$} edge[post](35);
%     \node[place](15)[left of=13, xshift=-20mm]{$\textrm{tpr}_{1}, \textrm{xi}_{1}$} edge[post](31);
% 	\node[empty](01)[above of=11]{};
%     \node[place](02)[above of=12]{$(\zeta, \eta)_{\textrm{tpr}}, (\zeta, \eta)_{\textrm{xi}}$} edge[post](12);
%     \node[place](03)[above of=13]{$(\zeta, \eta)_{\textrm{tpr}}, (\zeta, \eta)_{\textrm{xi}}$} edge[post](13);
%     \node[place](04)[above of=14]{$(\zeta, \eta)_{\textrm{tpr}}, (\zeta, \eta)_{\textrm{xi}}$} edge[post](14);
%     \node[place](05)[above of=15]{$(\zeta, \eta)_{\textrm{tpr}}, (\zeta, \eta)_{\textrm{xi}}$} edge[post](15);
% 	\end{scope}
% 	\end{tikzpicture}
%     \caption{Graphical representation of the POPP-Beta.}
% 	\label{fig:gm_popp_beta}
% \end{figure*}

%!TEX root = ../bare_jrnl.tex

\subsection{The POPP-Dirichlet}
\label{subsec:popd}

The C-POPP model requires the true positive rate $P^+$ and false positive rate $P^-$ to be specified in advanced in estimating the parameter $\lambda$ of a Poisson process. These are an extension of $\tau$ and $\xi$ where the rates provide a probability for a particular combination of binary detections coming from each sensor given the true event as shown in Eq. \ref{eq:joint_sensor_model_positive_event} and Eq. \ref{eq:joint_sensor_model_negative_event}.

To construct an observation model of $P^+$ and $P^-$, one needs to have both detections and the corresponding actual (non-)events as ground truth. Pre-processing involving expert interventions is typically required before the detections and their corresponding ground truth can be further used. Similarly to the POPP model, the C-POPP model requires the observation model to be accurate to avoid the posterior over $\lambda$ drifting away from the true posterior. If attaining an accurate observation model for the POPP model is a problem, then this becomes more challenging in the case of C-POPP model. This is because the training data needed to construct an observation model grows by a factor of two for each sensor involved.       

Analogously to the extension from the POPP model to the POPP-Beta model, we can expand the C-POPP observation model. In this case the observation models ($P^+$ and $P^-$) will follow Dirichlet distributions. The Dirichlet distribution is an appropriate distribution since $P^+$ and $P^-$ are the probabilities of categorical distributions which set the probabilities of multinomial distributions in Eq. \ref{eq:codependent_sensor_likelihood} and Dirichlet distributions provide a family of conjugate prior probability distributions for the multinomial distribution. The Dirichlet-multinomial conjugacy leads to an analytically tractable compound distribution which is called the Dirichlet-multinomial distribution, where the $\mathbf{p} = (p_1, \ldots, p_r)$ parameter in the multinomial distribution $Mult(\mathbf{d} \mid c, \mathbf{p})$ is randomly drawn from a Dirichlet distribution $Dir(\mathbf{p} \mid \mathbf{\zeta})$. 
\begin{equation}
	\label{eq:beta_binomial_revisit}
	\begin{tabular}{r@{ = }l}
        $P(\mathbf{d} \mid c, \mathbf{\zeta})$ & $\displaystyle\int P(\mathbf{d} \mid c, \mathbf{p}) ~ P(\mathbf{p} \mid \mathbf{\zeta}) ~d\mathbb S_r$ \\ [2ex]
        & $\displaystyle\int Mult(\mathbf{d} \mid c, \mathbf{p}) ~ Dir(\mathbf{p} \mid \mathbf{\zeta}) ~d\mathbb S_r$ \\ [2ex]
        & $DM((d_1, \ldots, d_r) \mid c, (\zeta_1, \ldots, \zeta_r))$
	\end{tabular}
\end{equation}
\noindent with $\mathbf{d} = (d_1, \ldots, d_r)$, $\mathbf{\zeta} = (\zeta_1, \ldots, \zeta_r)$, and $d\mathbb S_r$ denotes integrating $\mathbf{p}$ with respect to the $(r - 1)$ simplex\footnote{The support of the Dirichlet distribution is the $(r - 1)$-dimensional simplex $\mathbb S_r$; that is, all $r$ dimensional vectors which form a valid probability distribution}.

Given $m$ sensors, an observation model is now represented as two Dirichlet distributions: $Dir(P^+ \mid \mathbf{\zeta^+})$, and $Dir(P^- \mid \mathbf{\zeta^-})$ with $\mathbf{\zeta^+} = (\zeta^+_0, \ldots, \zeta^+_{(m^2)-1})$ and $\mathbf{\zeta^-} = (\zeta^-_0, \ldots, \zeta^-_{(m^2)-1})$. $\mathbf{\zeta^+}$ and $\mathbf{\zeta^-}$ set the overall shape of the Dirichlet priors, with each $\zeta_q$ term counting the number of times that particular combination of sensor detections were produced given a positive ($\mathbf{\zeta^+}$, $e=1$) or negative ($\mathbf{\zeta^-}$, $e=0$) detection.

Given a joint sensor model where its elements follow a Dirichlet density and several Dirichlet-multinomial distributions, which provide an unconditional distribution of $\mathbf{d}$, we replace Eq. \ref{eq:codependent_sensor_likelihood} with:  
\begin{equation}
	\label{eq:joint_dirichlet_multinomial_distribution}
    \begin{tabular}{r@{=}l}
		$P(\mathbf{D} \mid c)$ & $\sum\limits_{(\mathbf{e}^+, \mathbf{e}^-) \in \Sigma_c} DM(\mathbf{g}^+ \mid c, \mathbf{\zeta^+}) ~ DM(\mathbf{g}^- \mid (l - c), \mathbf{\zeta^-})$
	\end{tabular}
\end{equation}
\noindent with $\Sigma_c$ and $\mathbf{D}$ as defined in Section~\ref{subsec:cpop}.
% With a joint sensor model following the Dirichlet density, which is conjugated with multinomial distributions into a posterior predictive distribution shown in Eq. \ref{eq:joint_dirichlet_multinomial_distribution}, a graphical model is shown in Figure \ref{fig:gm_popp_dirichlet}.

The difference between the C-POPP model and the POPP-Dirichlet lies only in Eq. \ref{eq:codependent_sensor_likelihood} being replaced by \ref{eq:joint_dirichlet_multinomial_distribution}. However, given a certain Dirichlet prior, and limited training data for the sensor model, the POPP-Dirichlet is expected to be more conservative in estimating the posterior $P(\lambda \mid \mathbf{s})$ over $\lambda$ than the C-POPP model.

% \begin{figure}[t!]
% 	\centering
% 	\begin{tikzpicture}
% 	\tikzstyle{place}=[rectangle,draw=blue,thick,minimum size=5 mm]
% 	\tikzstyle{every label}=[black]
% 	\begin{scope}
%     \node[place](51)[xshift=30mm]{$(\zeta^+_0, \ldots, \zeta^+_{(m^2)-1})$};
%     \node[place](52)[right of=51, xshift=30mm]{$(\zeta^-_0, \ldots, \zeta^-_{(m^2)-1})$};
%     \node[place](41)[above of=51, yshift=3mm]{$(E^+_0, \ldots, E^+_{(m^2)-1})$} edge[pre](51);
%     \node[place](42)[above of=52, yshift=3mm]{$(E^-_0, \ldots, E^-_{(m^2)-1})$} edge[pre](52);
%     \node[place](31)[above of=41, xshift=-30mm, yshift=3mm]{$S_{1i}$} edge[pre](41) edge[pre](42);
% 	\node[place](32)[right of=31, xshift=20mm]{$S_{2i}$} edge[pre](41) edge[pre](42);
% 	\node[place](33)[right of=32, xshift=10mm]{$\ldots$} edge[pre](41) edge[pre](42);
% 	\node[place](34)[right of=33, xshift=15mm]{$S_{(m-1)i}$} edge[pre](41) edge[pre](42);
% 	\node[place](35)[right of=34, xshift=22mm]{$S_{mi}$} edge[pre](41) edge[pre](42);
% 	\node[place](21)[above of=33]{$X_i$} edge[post](31) edge[post](32) edge[post](33) edge[post](34) edge[post](35);
% 	\node[place](11)[above of=21]{$\lambda$} edge[post](21);
% 	% \node[place](01)[above of=11]{$\alpha, \beta$} edge[post](11);
% 	\end{scope}
% 	\end{tikzpicture}
% 	\caption{Graphical representation of the POPP-Dirichlet.}
% 	\label{fig:gm_popp_dirichlet}
% \end{figure}

\section{Evaluation on Synthetic Data}
\label{sec:evasim}

Evaluation on synthetic data is used to establish the benefit of the POPP and its variations over the FOPP in estimating the arrival rate $\lambda$ of a Poisson process. With synthetic data, sensor reliability can be controlled, and the true $\lambda$ and the true counts $c_i$ are known in advance for each sample. Here, an evaluation and a comparison of the POPP models to the FOPP model are conducted with two imaginary unreliable sensors and simulated datasets. The switching filter is chosen as a filter for all POPP models for this evaluation.

In each experiment, the sensor model (for the POPP and POPP-Beta models) and joint sensor model (for the C-POPP and POPP-Dirichlet models) is first built based on the sampled counts $c_1, \ldots, c_n$ from a Poisson process $P(c ; \lambda'=3)$ and the corresponding sensor readings $\protect\overrightarrow{s_1} \ldots \protect\overrightarrow{s_{n}}$. Then another set of counts $c_1, \ldots, c_{144}$ is sampled from the same process. These counts were then fed to simulated sensors that counted unreliably, producing sensor readings $\protect\overrightarrow{s_1} \ldots \protect\overrightarrow{s_{144}}$. A recursive update, then, takes place on $P(\lambda ; \overrightarrow{s_i})$ using the switching filter.

Two different sample sizes used to build the (joint) sensor model were chosen: a small number of samples, and a large number of samples. A small number of samples build an erroneous (joint) sensor model. In the POPP-Dirichlet and the POPP-Beta models, a small number of samples creates a loose Dirichlet and beta densities. Conversely, a large number of samples creates a low variance, and thus reliable, (joint) sensor model by tightening the Dirichlet prior of the POPP-Dirichlet model and the Beta prior of the POPP-Beta model. 120 samples were set for the small sample size, and 2880 samples were set for the large sample size.

Different correlations between two sensors were tested: positive correlation, negative correlation, and no correlation. These correlations are aimed to show the benefit of the C-POPP and POPP-Dirichlet models over any other model including the POPP and the POPP-Beta models. One should note that under the POPP and POPP-Beta models, sensors are assumed to be uncorrelated.  

\begin{figure}[t!]
	\centering
	\includegraphics[width=0.5\textwidth]{./figures/tjpr_comparison_120.png}
    \caption{The RMSE of posterior estimates of $\lambda$ for the POPP-Dirichlet and other POPP models with 120 sample data used to build the (joint) sensor model with variation on $\mathcal{E^+}$. Each trial consisted of a stream of $\protect\overrightarrow{s_1} \ldots \protect\overrightarrow{s_{144}}$ samples to update $P(\lambda ; \protect\overrightarrow{s_i})$. Accuracies of MAP estimates are shown in the top panel, accuracies of expectation of the posterior in the bottom panel. Each data point is an average of 30 trials. Standard errors are shown.} 
	\label{fig:tjpr_comparison_120}
\end{figure}

\begin{figure}[t!]
	\centering
	\includegraphics[width=0.5\textwidth]{./figures/tjpr_comparison_2880.png}
    \caption{The RMSE of posterior estimates of $\lambda$ for the POPP-Dirichlet and other POPP models with 2880 sample data used to build the (joint) sensor model with variation on $\mathcal{E^+}$. Each trial consisted of a stream of $\protect\overrightarrow{s_1} \ldots \protect\overrightarrow{s_{144}}$ samples to update $P(\lambda ; \protect\overrightarrow{s_i})$. Accuracies of MAP estimates are in the top panel, accuracies of expectation of the posterior in the bottom panel. Each data point is an average of 30 trials. Standard errors are shown.} 
	\label{fig:tjpr_comparison_2880}
\end{figure}

For each correlation type, a further variation to different levels of sensor unreliability was considered. First, two variations were made to the true joint positive rate $\mathcal{E^+}$ (TJPR), while fixing the true joint negative rate $\mathcal{E^-}$ (TJNR) on each type of correlation. This includes:
\begin{itemize}
    \item $P_{jnt}(d_{1k}=1, d_{2k}=1 ; e_k=1) = 0.1, P_{jnt}(d_{1k}=0, d_{2k}=0 ; e_k=1) = 0.9$ ($\mathcal{E^+}$ with low positive correlation);
    \item $P_{jnt}(d_{1k}=1, d_{2k}=1 ; e_k=1) = 0.9, P_{jnt}(d_{1k}=0, d_{2k}=0 ; e_k=1) = 0.1$ ($\mathcal{E^+}$ with high positive correlation);
    \item $P_{jnt}(d_{1k}=1, d_{2k}=0 ; e_k=1) = 0.05, P_{jnt}(d_{1k}=0, d_{2k}=1 ; e_k=1) = 0.05, P_{jnt}(d_{1k}=0, d_{2k}=0 ; e_k=1) = 0.9$ ($\mathcal{E^+}$ with low negative correlation);
    \item $P_{jnt}(d_{1k}=1, d_{2k}=0 ; e_k=1) = 0.45, P_{jnt}(d_{1k}=0, d_{2k}=1 ; e_k=1) = 0.45, P_{jnt}(d_{1k}=0, d_{2k}=0 ; e_k=1) = 0.1$ ($\mathcal{E^+}$ with high negative correlation);
    \item $P_{jnt}(d_{1k}=1, d_{2k}=0 ; e_k=1) = 0.033, P_{jnt}(d_{1k}=0, d_{2k}=1 ; e_k=1) = 0.033, P_{jnt}(d_{1k}=1, d_{2k}=1 ; e_k=1) = 0.033, P_{jnt}(d_{1k}=0, d_{2k}=0 ; e_k=1) = 0.901$ ($\mathcal{E^+}$ with no correlation -- Similar to a sensor model with TPR = 0.066);
    \item $P_{jnt}(d_{1k}=1, d_{2k}=0 ; e_k=1) = 0.3, P_{jnt}(d_{1k}=0, d_{2k}=1 ; e_k=1) = 0.3, P_{jnt}(d_{1k}=1, d_{2k}=1 ; e_k=1) = 0.3, P_{jnt}(d_{1k}=0, d_{2k}=0 ; e_k=1) = 0.1$ ($\mathcal{E^+}$ with no correlation -- Similar to a sensor model with TPR = 0.6).
\end{itemize}

Second, two variations to the true joint negative rate $\mathcal{E^-}$ (TJNR), while fixing the true joint positive rate $\mathcal{E^+}$ (TJPR) on each type of correlation. This includes: 
\begin{itemize}
    \item $P_{jnt}(d_{1k}=1, d_{2k}=1 ; e_k=0) = 0.1, P_{jnt}(d_{1k}=0, d_{2k}=0 ; e_k=0) = 0.9$ ($\mathcal{E^-}$ with low positive correlation);
    \item $P_{jnt}(d_{1k}=1, d_{2k}=1 ; e_k=0) = 0.9, P_{jnt}(d_{1k}=0, d_{2k}=0 ; e_k=0) = 0.1$ ($\mathcal{E^-}$ with high positive correlation);
    \item $P_{jnt}(d_{1k}=1, d_{2k}=0 ; e_k=0) = 0.05, P_{jnt}(d_{1k}=0, d_{2k}=1 ; e_k=0) = 0.05, P_{jnt}(d_{1k}=0, d_{2k}=0 ; e_k=0) = 0.9$ ($\mathcal{E^-}$ with low negative correlation);
    \item $P_{jnt}(d_{1k}=1, d_{2k}=0 ; e_k=0) = 0.45, P_{jnt}(d_{1k}=0, d_{2k}=1 ; e_k=0) = 0.45, P_{jnt}(d_{1k}=0, d_{2k}=0 ; e_k=0) = 0.1$ ($\mathcal{E^-}$ with high negative correlation);
    \item $P_{jnt}(d_{1k}=1, d_{2k}=0 ; e_k=0) = 0.033, P_{jnt}(d_{1k}=0, d_{2k}=1 ; e_k=0) = 0.033, P_{jnt}(d_{1k}=1, d_{2k}=1 ; e_k=0) = 0.033, P_{jnt}(d_{1k}=0, d_{2k}=0 ; e_k=1) = 0.901$ ($\mathcal{E^-}$ with no correlation -- Similar to a sensor model with low TNR);
    \item $P_{jnt}(d_{1k}=1, d_{2k}=0 ; e_k=0) = 0.3, P_{jnt}(d_{1k}=0, d_{2k}=1 ; e_k=0) = 0.3, P_{jnt}(d_{1k}=1, d_{2k}=1 ; e_k=0) = 0.3, P_{jnt}(d_{1k}=0, d_{2k}=0 ; e_k=1) = 0.1$ ($\mathcal{E^-}$ with no correlation -- Similar to a sensor model with moderate TNR).
\end{itemize}

\begin{figure}[t!]
	\centering
	\includegraphics[width=0.5\textwidth]{./figures/tjnr_comparison_120.png}
    \caption{The RMSE of posterior estimates of $\lambda$ for the POPP-Dirichlet and other POPP models with 120 sample data used to build the (joint) sensor model with variation in $\mathcal{E^-}$. Each trial consisted of a stream of $\protect\overrightarrow{s_1} \ldots \protect\overrightarrow{s_{144}}$ samples to update $P(\lambda ; \protect\overrightarrow{s_i})$. Accuracies of MAP estimates are  in the top panel, accuracies of the expectation of the posterior in the bottom panel. Each data point is an average of 30 trials. Standard errors are shown.} 
	\label{fig:tjnr_comparison_120}
\end{figure}

\begin{figure}[t!]
	\centering
	\includegraphics[width=0.5\textwidth]{./figures/tjnr_comparison_2880.png}
    \caption{The RMSE of posterior estimates of $\lambda$ for the POPP-Dirichlet and other POPP models with 2880 sample data used to build the (joint) sensor model with variation in $\mathcal{E^-}$. Each trial consisted of a stream of $\protect\overrightarrow{s_1} \ldots \protect\overrightarrow{s_{144}}$ samples to update $P(\lambda ; \protect\overrightarrow{s_i})$. Accuracies of MAP estimates are in the top panel, accuracies of the expectation of the posterior in the bottom panel. Each data point is an average of 30 trials. Standard errors are shown.} 
	\label{fig:tjnr_comparison_2880}
\end{figure}

The performance of all POPP models and the FOPP model were also assessed by comparing the Jensen-Shannon distance. Jensen-Shannon distance is a unit of measurement used in Jensen-Shannon divergence. The Jensen-Shannon divergence is a method of measuring the similarity between two probability distributions. Unlike KL-divergence, Jensen-Shannon divergence is a symmetrized divergence where $D_{JS}(P \parallel Q)$ is equal to $D_{JS}(Q \parallel P)$.

\begin{figure}[t!]
	\centering
	\includegraphics[width=0.5\textwidth]{./figures/tjpr_comparison_120_kl.png}
	\caption{The Jensen-Shannon distance of posterior estimates of $\lambda$ for the POPP-Dirichlet and other POPP models with 120 sample data used to build the (joint) sensor model with variation on $\mathcal{E^+}$. Each trial consisted of a stream of $\protect\overrightarrow{s_1} \ldots \protect\overrightarrow{s_{144}}$ samples to update $P_G(\lambda \mid \protect\overrightarrow{s_i})$. Each data point is an average of 30 trials. Standard errors are shown.} 
	\label{fig:tjpr_comparison_120_kl}
\end{figure}

\begin{figure}[t!]
	\centering
	\includegraphics[width=0.5\textwidth]{./figures/tjpr_comparison_2880_kl.png}
	\caption{The Jensen-Shannon distance of posterior estimates of $\lambda$ for the POPP-Dirichlet and other POPP models with 2880 sample data used to build the (joint) sensor model with variation on $\mathcal{E^+}$. Each trial consisted of a stream of $\protect\overrightarrow{s_1} \ldots \protect\overrightarrow{s_{144}}$ samples to update $P_G(\lambda \mid \protect\overrightarrow{s_i})$. Each data point is an average of 30 trials. Standard errors are shown.} 
	\label{fig:tpjr_comparison_2880_kl}
\end{figure}

\begin{figure}[t!]
	\centering
	\includegraphics[width=0.5\textwidth]{./figures/tjnr_comparison_120_kl.png}
	\caption{The Jensen-Shannon distance of posterior estimates of $\lambda$ for the POPP-Dirichlet and other POPP models with 120 sample data used to build the (joint) sensor model with variation on $\mathcal{E^-}$. Each trial consisted of a stream of $\protect\overrightarrow{s_1} \ldots \protect\overrightarrow{s_{144}}$ samples to update $P_G(\lambda \mid \protect\overrightarrow{s_i})$. Each data point is an average of 30 trials. Standard errors are shown.} 
	\label{fig:tjnr_comparison_120_kl}
\end{figure}

\begin{figure}[t!]
	\centering
	\includegraphics[width=0.5\textwidth]{./figures/tjnr_comparison_2880_kl.png}
	\caption{The Jensen-Shannon distance of posterior estimates of $\lambda$ for the POPP-Dirichlet and other POPP models with 2880 sample data used to build the (joint) sensor model with variation on $\mathcal{E^-}$. Each trial consisted of a stream of $\protect\overrightarrow{s_1} \ldots \protect\overrightarrow{s_{144}}$ samples to update $P_G(\lambda \mid \protect\overrightarrow{s_i})$. Each data point is an average of 30 trials. Standard errors are shown.} 
	\label{fig:tjnr_comparison_2880_kl}
\end{figure}

%!TEX root = ../bare_jrnl.tex

\section{Evaluation on Aggregate Human Occupancy Behaviour Dataset}
\label{sec:evareal}

We also investigated the performance and practicality of the POPP model and its extensions using switching filters on a large, real world data set.
 % Performance comparison against the FOPP model is now included in this evaluation.
% 
This dataset was gathered from an office building in which a mobile robot counts the number of people passing by, as it patrols (see Figure~\ref{fig:map_popp_independent_test} for the map). The data set contains a time series of counts from three different automated person detectors \cite{dondrup2015real}. These use laser, depth camera and RGB information. We refer to them respectively as the leg detector (LD), upper body detector (UBD), and change detector (CD). Each returns a sensed count of the number of people it detected in each 10 minute interval during the day. These detectors are unreliable, as can be seen from Figure~\ref{fig:single_sensor_rate_transformation}, which shows examples of correct and incorrect detections.

\begin{figure}[t]
	\centering
	\includegraphics[width=0.95\columnwidth]{./figures/map_popp.png}
	\caption{The office building in which the robot gathered data. Areas are bounded by imaginary lines.}
	\label{fig:map_popp_independent_test}
\end{figure}

\begin{figure}[t]
	\centering
	\includegraphics[width=0.95\columnwidth]{./figures/sensor_images.png}
	\caption{Correct and incorrect detections (and non-detections) from different regions in the environment for each sensor. Top row: change detector. Middle row: upper body detector. Bottom row: leg detector. Detections are marked with 2D or 3D bounding boxes.}
	\label{fig:single_sensor_rate_transformation}
\end{figure}

By comparing the ground truth with the detections made by sensors, we computed a sensor model for each region. An average of the sensor models across all the regions can be seen in Table \ref{table:sensor_model_popp_beta}. Although the robot operated 24/7, the sensor models were built using the data collected from 10am-8pm, there being negligible detections outside these times. From a 69-day deployment of the mobile robot, we obtained 48 days of usable observations. We specified a time interval for each Poisson distribution of 10 minutes, and recorded both the true counts and the detections made by each sensor in each interval. We assumed that the underlying process in each region to be a periodic Poisson process in which there is a one-day periodicity cycle, i.e. $\lambda(t) = \lambda(t + \Delta)$ with $\Delta = 24 * 60$ (minutes), in the process. This means that the expected number of people turning up on each day at a particular time of the day is expected to be the same across the 48 days of observations. We then estimated the true parameter $\lambda(t)$ of the Poisson distribution at $t$ interval by running a FOPP model on the true counts within each $t$ interval. We use this estimate of $\lambda(t)$ from the true counts as the target which the POPP models must estimate from the sensed counts.

The different POPP models rely on sensor models that must be calculated from a confusion matrix relating true counts and the different sensor counts. To separate the training and testing data we performed four fold cross-validation with data splits being on whole days, i.e., we used 12 days of data as a training set for a sensor model and then used the remaining 36 days of data as a test set on which to test the inferences made by each model from the sensor counts.

\begin{table}[t]
	\centering
	\caption{Averaged sensor model across all areas trained from 48 days of data.}
	\label{table:sensor_model_popp_beta}
	\begin{tabular}{lccc}
		\noalign{\hrule height 1.1pt}\noalign{\smallskip}
		Sensor & True Positive & True Negative \\
		\noalign{\smallskip}\hline\noalign{\smallskip}
		Leg Detector & 0.387 & 0.951 \\
		Upper Body Detector & 0.356 & 0.882 \\
		Change Detector & 0.731 & 0.900 \\ 
		\noalign{\hrule height 1.1pt}\noalign{\smallskip}
	\end{tabular}
\end{table}

For the 36 days of test data the different models each made predictions of the $\lambda(t)$ parameter of the Poisson. Given this, we recorded (1) a distance metric method using the RMSE of the MAP hypothesis of each model posterior distribution over $\lambda(t)$ to the true $\lambda'(t)$ and (2) a free-distance metric method using the Jensen-Shannon distance between the posterior distribution $P(\lambda(t) ; \overrightarrow{s_i})$ and the distribution of the true $\lambda'(t)$ generated from the FOPP model on true counts. Using these metrics, we compared the performance of all POPP models (again estimated using the switching filter from~\cite{jovan18a}) to the standard Bayes' filter arising from the FOPP model. The uncorrected estimate $\lambda(t)$ according to the FOPP model was estimated only from the change detector count data since the change detector is the most reliable detector among three detectors available in the robot as shown in Table \ref{table:sensor_model_popp_beta}.

Figure \ref{fig:fopp_popp_popb_npop_popd_rmse_evo} and \ref{fig:fopp_popp_popb_npop_popd_kl_evo} show the accuracy comparison between all POPP models and the standard FOPP model over time. It can be seen that all models become more and more accurate towards the true $\lambda'(t)$ as days pass. All POPP models show more accuracy over the standard FOPP model. The $\lambda(t)$ estimate produced by the POPP-Dirichlet model is more accurate than the ones produced by the standard POPP model and the POPP-Beta model. However, the estimate is not always more accurate compared to the one produced by the C-POPP model. 

As the POPP-Dirichlet model is more conservative in estimating the parameter $\lambda(t)$ than the C-POPP model, the estimate moves rather slowly towards the true $\lambda'(t)$. This is seen in Figure \ref{fig:fopp_popp_popb_npop_popd_kl_evo}.
By the third day, the POPP-Dirichlet model outperformed the POPP and the POPP-Beta, and the C-POPP models in terms of accuracy. However, the accuracy gap between the C-POPP model and the POPP-Dirichlet model became smaller and by the 36th day, where the C-POPP model outperformed the POPP-Dirichlet by a small margin.

\begin{figure}[t!]
	\centering
	\includegraphics[width=0.95\columnwidth]{./figures/fopp_popp_popb_npop_popd_rmse_evo.png}
	\caption{The RMSE evolution of periodic Poisson processes with POPP, POPP-Beta, C-POPP, POPP-Dirichlet and FOPP filters from day 3 to day 36, averaged across all regions. Standard error is shown.}
	\label{fig:fopp_popp_popb_npop_popd_rmse_evo}
\end{figure}

\begin{figure}[t!]
	\centering
	\includegraphics[width=0.95\columnwidth]{./figures/fopp_popp_popb_npop_popd_kl_evo.png}
	\caption{The Jensen-Shannon distance evolution of the FOPP, the POPP, the POPP-Beta, the C-POPP, and the POPP-Dirichlet filters in periodic Poisson processes from day 3 to day 36 in a 3-day interval, averaged across all regions. Standard error is shown.}
	\label{fig:fopp_popp_popb_npop_popd_kl_evo}
\end{figure}

%Similar to the C-POPP model, the POPP-Dirichlet model is able to cope and overcome the problems with limited sample data both for building the joint sensor model and estimating the $\lambda(t_i, t_j)$. In many regions, the POPP-Dirichlet managed to show better estimates as well as more similar distributions than the POPP, the POPP-Beta, and the FOPP models. However, the POPP-Dirichlet filter falls behind both in accuracy (RMSE) and distribution similarity compared to the C-POPP model. This is attributed to the POPP-Dirichlet conservative way in estimating the parameter $\lambda(t_i, t_j)$ compared to the C-POPP model.
%!TEX root = ../bare_jrnl.tex

\section{Exploring for Human Activities}
\label{sec:exploration}

So far, the paper has focused on Bayesian methods for inferring a belief state about the spatio-temporal patterns of human occupancy from unreliable sensors. Given such a belief state a robot may plan how to actively explore to acquire new information so as to complete a task~\cite{hanheide2017robot, sridharan2019reba}. Here, the robot uses predicted counts from the belief state to \emph{explore} so as to detect human activities with increasing efficiency. 

Specifically, the robot's choice is whether to explore new region-time combinations or to exploit region-time combinations that are known to yield a high number of activities. This an instance of an \emph{exploration-exploitation} problem. Exploration-exploitation problems arise whenever an agent lacks an adequate model of the process it must control. At each moment, the agent chooses either to explore so as to improve the model or to exploit the existing model so as to maximise immediate performance. 
% In each time the robot has a choice between many actions, each of which both explores and exploits a certain place, but to varying degrees. As its goal is to maximise the reward gathered–as in getting as much data on human activities (by seeing as many humans) as possible–given its limited operational life, it is preferable to have a policy that is as near optimal as possible.

While exploration-exploitation problems in reinforcement learning, are typically intractable, there are well known, fast to compute, approximations~\cite{wyatt1998exploration, 1413255, AUDIBERT20091876}. One such approach is to use the upper bound of a probability distribution over the quantity being maximised. This causes the decision-making agent to exploit high-scoring, certain estimates, and explore highly uncertain estimates. In our robot exploration, for example, when the robot visits a place, it can be because the place either actually has high number of people (\textit{exploitation}) or potentially has high number of people (\textit{exploration}). In our case we use an upper bound on the arrival rate ($\lambda$) of a Poisson process ($\lambda_{UB}$) to choose the region for the robot to visit next. The upper bound of the probability interval of the arrival rate of a Poisson process is calculated as follows:

\begin{equation}
	\label{eq:upper_bound_exploration}
	\begin{tabular}{r@{ = }l}
	$\lambda_{UB}(t_i, t_j)$ & $\displaystyle \int_{t_i}^{t_j} CDF^{-1}(\% = 0.95 \mid \alpha_t, \beta_t)~dt$\\ [1ex]
	\end{tabular}
\end{equation}

\noindent with $\lambda_{UB}(t_i, t_j)$ as the upper bound of $\lambda$ within time $t_i$ and $t_j$, $i, j \in \{1, \ldots, \Delta\}$, and $CDF^{-1}$ as the inverse of the cumulative density function of a Gamma distribution. Given the upper bounds $\lambda^{r}_{UB}(t_i, t_j)$ for each region $r$ from the set of all regions $R$, the region to be visited between time $t_i$ and $t_j$ is chosen by:

\begin{equation}
\label{eq:choosing_place}
\underset{r \in \mathcal R}{\arg\max}~\lambda^{r}_{UB}(t_i, t_j)
\end{equation}
\noindent Figure~\ref{fig:map_vs_ub} depicts a comparison between the MAP hypothesis estimate and the upper bound estimate of a Poisson process.

\begin{figure}[t!]
	\centering
	\includegraphics[width=0.5\textwidth]{./figures/map_vs_ub.png}
	\caption{A spectral Poisson process of region 9 (see Figure \ref{fig:map_popp_independent_test}) represented by its MAP hypothesis (blue line) and its upper bound of the probability interval (red line).}
	\label{fig:map_vs_ub}
\end{figure}


To tie the estimate of a particular Poisson process over a time interval to data collected previously, as in Section~\ref{sec:evareal} we assume that human presence in each region follows a \emph{periodic} Poisson process with daily periodicity. This allows us to regularise, and fill missing data, across the point estimates of upper bounds using methods based on the Fourier transform.  This exploits assumptions and algorithms introduced in our prior work. In particular, the series of upper bounds $\lambda_{UB}(t_i, t_j)$ are encoded and extracted via spectral analysis with the $l$-AAM technique described in~\cite{jovan_iros16}. The plot in Fig.~\ref{fig:map_vs_ub} shows how a spectral Poisson process look like, i.e., the effects of the spectral processing on a periodic Poisson process. Algorithm 2 depicts the process of computing the upper bound of a Poisson process and applying spectral analysis to it. We use this approach with upper bounds produced by our previously presented estimators: FOPP, POPP, and POPP-Beta. C-POPP and POPP-Dirichlet estimators are excluded in our experiments due to a need to limit experimental time to ~45 days to keep building use conditions that were broadly the same.\footnote{The experiments were conducted during a single academic semester within a university building.}

\begin{figure}[t!]
	\begin{center}
		\begin{tabular*}{0.5\textwidth}{l @{\extracolsep{\fill}}}
			\hline
			\textbf{l-AAM} \textrm{\cite{jovan_iros16}} \\
			\hline
			\textbf{Input:} $x_1, \ldots, x_n$: input signal, \\
			\hspace{0.3cm} total: maximum total frequency \\
			\textbf{Output:} $\mathcal S$: a collection of $(s, p, f)$ \\
			\textbf{Procedure:}\\
			\hspace{0.3cm} 1. Init. k $\leftarrow$ 0 \\
			\hspace{0.3cm} // Get frequency $0$ with Discrete Fourier Transform \\
			\hspace{0.3cm} 2. $[s, p, f] \leftarrow DFT(x_1, \ldots, x_n)[0]$\\
			\hspace{0.3cm} 3. $\mathcal S[k] \leftarrow [s, p, f]$ \\
			\hspace{0.3cm} 4. Repeat until k $>$ total \\
			\hspace{0.7cm} $\bullet ~ k \leftarrow k + 1$ \\
			\hspace{0.7cm} // Get the frequency with the highest amplitude \\
			\hspace{0.7cm} $\bullet ~ [s, p, f] \leftarrow \argmax_s DFT(x_1, \ldots, x_n)$ \\
			\hspace{0.7cm} // Update $\mathcal S$ with frequency $f$ \\
			\hspace{0.7cm} $\bullet$ if $f \in \mathcal S$, $[s', p', f'] \leftarrow \mathcal S[k', f'=f]$ \\
			\hspace{2.5cm} $s \leftarrow s + s'$; $p \leftarrow p + p'$ \\ 
			\hspace{0.7cm} $\bullet$ $\mathcal S[k] \leftarrow [s, p, f]$ \\
			\hspace{0.7cm} // Create a cosine signal from $f$ \\
			\hspace{0.7cm} $\bullet ~ x'_1, \ldots, x'_n \leftarrow s * ~ cos(2 \pi * f + p)$ \\
			\hspace{0.7cm} // Subtract current $x_1, \ldots, x_n$ with the cosine signal \\
			\hspace{0.7cm} $\bullet ~ x_1, \ldots, x_n \leftarrow x_1, \ldots, x_n - x'_1, \ldots, x'_n$ \\
			\hline
		\end{tabular*}	
	\end{center}
\end{figure}

\begin{figure}[t!]
	\begin{center}
		\begin{tabular*}{0.5\textwidth}{l @{\extracolsep{\fill}}}
			\hline
			\textbf{Algorithm 2} \textit{Upper Bound} \\
			\hline
			\textbf{Input:} $(\alpha_1, \beta_1), \ldots, (\alpha_n, \beta_n)$: Poisson process \\
			\textbf{Output:} $\lambda^{ub}_1, \ldots, \lambda^{ub}_n$: upper bound \\
			\textbf{Procedure:}\\
			\hspace{0.3cm} 1. Init. k $\leftarrow$ 1, m $\leftarrow$ $\eta$ \\
			\hspace{0.3cm} 2. Repeat until k $>$ n \\
			\hspace{0.7cm} $\bullet ~ k \leftarrow k + 1$ \\
			\hspace{0.7cm} // Get the upper bound \\
			\hspace{0.7cm} $\bullet ~ \lambda_k \leftarrow CDF(0.95, \alpha_k, \beta_k)$ \\
			\hspace{0.3cm} // Transform $\lambda_1, \ldots, \lambda_n$ to with $l$-AAM \\
			\hspace{0.3cm} 3. $\mathcal S$ $\leftarrow$ \textbf{l-AAM}($\lambda_1, \ldots, \lambda_n$, m) \\
			\hspace{0.3cm} 5. Init. k $\leftarrow$ 0,  $\lambda^{ub}_1, \ldots, \lambda^{ub}_n \leftarrow (0, \ldots, 0)$ \\
			\hspace{0.3cm} 4. Repeat until k $>$ m \\
			\hspace{0.7cm} // Create a cosine signal from $\mathcal S[k]$ \\
			\hspace{0.7cm} $\bullet ~ [s, p, f] \leftarrow \mathcal S[k]$ \\
			\hspace{0.7cm} $\bullet ~ x_1, \ldots, x_n \leftarrow s * ~ cos(2 \pi * f + p)$ \\
			\hspace{0.7cm} // Add current $\lambda^{ub}_1, \ldots, \lambda^{ub}_n$ with the cosine signal \\
			\hspace{0.7cm} $\bullet ~ \lambda^{ub}_1, \ldots, \lambda^{ub}_n \leftarrow \lambda^{ub}_1, \ldots, \lambda^{ub}_n + x_1, \ldots, x_n$ \\
			\hline
		\end{tabular*}	
	\end{center}
\end{figure} 


\subsection*{Exploration Evaluation}

The dataset used in the previous section was collected by a mobile robot over 69 days of a real world trial. This robot was controlled by the exploration models described above. Due to hardware failures, sensor malfunctions and other external issues, only 48 days from the dataset were usable.

Three different exploration models were applied separately during three phases of the 69 days of the trial. All of these models used Eq.~\ref{eq:choosing_place} to create their exploration policies. For the first 27 day phase of the trial, the robot followed an exploration policy based on the FOPP model. This resulted in 18 days of data. From day 28 to day 47, the robot followed an exploration policy according to the POPP model. This resulted in 15 days of data. Finally, from day 48 onwards, the robot followed an exploration policy according to the POPP-Beta model. This also resulted in 15 days of data. 
% 
Such that all three models can be compared equally, in the following we also constrain the data available for for the FOPP model to the first 15 of its 18 days.
% 
% From this 18 days of data, the last 3 days were used to train the sensor model needed for both the POPP and the POPP-Beta models.
%For this comparison, the last 3 days which are part of the 18 days worth of data collected by following the FOPP exploration model are included in the POPP and the POPP-Beta exploration models. This is necessary to avoid the POPP and the POPP-Beta exploration models having an advantage over the FOPP exploration model since the POPP and the POPP-Beta need a training period to construct their sensor model. Moreover,
% 
We can compare the different exploration policies on the observations the robot made during the phase each policy was active. Due to the absence of information regarding occupancy in the places that the robot did not visit, only a comparison of the positive observations can be made. 

% As a note, a positive observation is a duration when the robot observes any activity during its visit to a particular area. 

\begin{figure}[t!]
	\centering
	\includegraphics[width=0.5\textwidth]{./figures/exploration_percentage_region.png}
	\caption{This graph shows the percentage of time that the robot observed activities when it was present in a region. It is a measure of how successful the robot's visit policy (choice of visit time and visit location) was in finding people. It presents results for for the FOPP, POPP and POPP-algorithms. %The activity exploration percentage across areas of the environment using three different exploration models (FOPP, POPP, POPP-Beta). The percentage shows the portion of time that the robot was observing activities.
	}
	\label{fig:exploration_percentage_region}
\end{figure}

%\begin{figure}[t!]
%	\centering
%	\includegraphics[width=0.5\textwidth]{./figures/exploration_improvement_ratio.png}
%	\caption{The improvement evolution of activity observation (in ratio) using three different exploration models. The average of the first three days of the number of observations on each exploration is used as the base (ratio 1.0). Black dots represent weekends the explorations were passing through.}
%	\label{fig:exploration_improvement_ratio}
%\end{figure}

%\begin{figure}[t!]
%	\centering
%	\includegraphics[width=0.5\textwidth]{./figures/exploration_number_people_across_days.png}
%	\caption{Average number of activities observed over days across regions. Black dots represent weekends the explorations were passing through.}
%	\label{fig:exploration_number_people_across_days}
%\end{figure}

\begin{figure}[t!]
	\centering
	\includegraphics[width=0.5\textwidth]{./figures/exploration_number_people_across_days_normalised.png}
	\caption{The improvement ratio of activity observations during each phase of the trial. The dash line indicates a baseline performance, i.e., no improvement in exploration over time.}
	\label{fig:exploration_improvement_ratio}
\end{figure}

Figure \ref{fig:exploration_percentage_region} shows the percentage of visits to each region which yielded a non-zero true count. As can be seen, the exploration policy produced by POPP-Beta has the highest proportion of such visits in many of the regions, followed by the exploration policy according to the POPP model. Recall that some regions, such as 4, 5, 6, and 7, are not densely populated with humans across time compared to other regions (such as 1, 2, 3, and 10). The POPP and POPP-Beta models, however, still managed to improve the percentage of positive observations. This shows that the models correctly predicted that people would be present in particular locations at particular times. One should note that region 6 contains vending machines which are often detected as a person by the upper body detector. This leads to the FOPP model planning to visit this particular location when no activity is taking place. The POPP and the POPP-Beta models were able to correct the miscounts occurring in region 6, providing a better estimate of the posterior over the arrival rate $\lambda$. This leads to models that better capture the true underlying process and thus support more accurate exploration-exploitation trade-offs.

% Figure \ref{fig:exploration_number_people_across_days} shows the average number of activities observed across regions on each day. 

During the first few days of each 15 day phase the robot primarily explores since each model initially has a highly uncertain estimate of $\lambda$. As more days of data are experienced the estimates increase in confidence and the robot starts to exploit this increased confidence by visiting locations which are likely to provide higher counts\footnote{Note that this change from exploration to exploitation occurs naturally and gradually in an upper bound-based model, and therefore the characterisation of the behaviour as exploring or exploiting is a \emph{post-hoc} justification.}.
% 
To allow us to produce a metric for a fair comparison across three models (FOPP, POPP, and POPP-Beta) deployed at different times (and thus experiencing different population dynamics), we look at the ratio between the expected observations made by a baseline policy and those made by our exploration policy in the same period. 
% 
To create the baseline total for each model we take the true counts experienced for its first three days then multiply these by five to give an expected total over 15 days (the number of days of data available to every model). This is the denominator in Eqn.~\ref{eq:metric}, where $s(n)$ is the (true) number of people observed on day $n$. This is used to divide the cumulative number of observations up to the current day:

\begin{equation}
	\label{eq:metric}
\hat{s}(n) = \frac{\displaystyle\sum_{i=1}^{n} s(i)}{\displaystyle\sum_{i=1}^{3} s(i) * 5}
\end{equation}

\noindent Given this, a $\hat{s}$ score of 1.0 on day 15 shows that people have been observed people at the rate of the baseline, i.e.\ the underlying model has failed to exploit additional data correctly. A result over 1.0 shows that the model has exploited the available data to observe people at a greater rate than in the first 3 days. 
% 
Figure \ref{fig:exploration_improvement_ratio} presents the cumulative normalised true counts of people observed by the robot across the three phases. This shows that exploration driven by the POPP and the POPP-Beta models improves the number of people observed during these phases. By the end of each of these two phases, the ratio is around 1.7. On the other hand, the FOPP showed a stable ratio around the baseline (1.0 at day 15), this means that the FOPP is not be able to improve the number of people observed over time. 
% 
Also note the general trend observed earlier that the approach which represents the uncertainty in the sensor models (POPP-Beta) initially out-perform less informed approach (POPP) until the latter has observed enough of the underlying process to compensate for training inaccuracies. 

%Unfortunately for the POPP-Beta, by looking at figure \ref{fig:exploration_improvement_ratio}, we can not conclude whether the POPP-Beta can improve the number of activities observed over time.
%We argue that this is because the exploration policy produced by the Specctral-POPP-Beta was running through several weekends. We assumed of daily periodicities for the non-homogeneous Poisson processes across regions, and the assumption was broken by running the exploration throughout weekends. This is because weekdays and weekend have different arrival rate $\lambda$. Figure \ref{fig:exploration_number_people_across_days} clearly shows how small the total activities observed during weekend compared to weekdays.
\section{Conclusion}
\label{sec:conclusion}

This work has been concerned with developing practical estimators for count data collected by an autonomous mobile robot, with unreliable perception algorithms. These count data represent the level of human activity in particular locations. This work extends the work in \cite{jovan18a} with several contributions:

\begin{itemize} 
    % \item A set of inference methods for the partially observable Poisson process (POPP) has been formulated. The POPP is a Poisson process which takes into account the unreliability of the sensors that count events. Unlike Bayesian estimation for a fully observable Poisson process (FOPP), obtaining the posterior is non-trivial, since there is no conjugate density for a POPP and the posterior has a number of elements that grow exponentially in the number of observed intervals. Two simple, tractable, approximations have been presented. These two approximations are combined in a switching filter, which enables efficient and accurate estimation of the posterior. A simulation study shows that these POPP filters correct the over- and under-counts produced by sensors.  
    \item Variations of the POPP filter are presented. The POPP-Beta filter extends the POPP filter in which the unreliability of the observation model is accounted for when estimations are built. The N-POPP filter extends the POPP filter by modelling the case when sensors are uncorrelated. The POPP-Dirichlet combines the POPP-Beta filter and the N-POPP filter to have the benefits of each correction. A simulation and observations taken by a robot, on a series of long deployments, show that each extension provides progressively more accurate estimates than the POPP filter.  
    \item Both posteriors from the Spectral-FOPP and two Spectral-POPP processes are used to drive exploration by a mobile robot for a series of two week deployments. An upper bound interval exploration method was used to solve the exploration-exploitation problem. After labelling by humans, this resulted in a labelled data set of six weeks of human activity levels. A simulated study has shown that the Spectral-FOPP and the Spectral-POPP filter improve on-point observation time significantly if strong periodic patterns underlying the human activities are present. 
\end{itemize}
        
        % Finally, the various POPP filters are compared to one another and to a FOPP estimator on this data set.

% which capture the regular structures of dynamic behaviours, especially humans,    
% 
% 
% Hence, any learning algorithm requires practical estimation which capture temporal structures of human activities.   
% 
% The adaptation requires practical estimation which capture the structure 
% 
% The robot will be able to demonstrate that it can recognise activities at various
% temporal scales, and infer or predict future activities based on its temporal model (e.g. it
% might go to the lounge because it has just seen the residents finishing their lunch). It
% will also be able to detect anomalies as sequences of very low likelihood data.
% 
% The robot only observes a limited portion of the space at any time, and so must actively plan to go to places to observe events.
% 
% learns dynamic behaviours of its surrounding while patroling around perimeters of a large area. 
% 
% Human activities follow predictable, repeating patterns that generate corresponding
% changes in space.
% 
% Our
% work will allow a robot to create a map of a building and its contents: not just walls but people,
% furniture and objects, all in a unified spatial-temporal representation that will allow a robot to
% respond robustly to the dynamics of its environment. In order to reason about the structure
% and purpose of these dynamics we will employ the spatial-temporal representations in support of
% activity recognition. This will allow the robot to detect, and exploit, patterns of human behaviour.
% 
% In our care scenario we will explore how a robot can support staff working with a small group of
% elderly patients in a nursing facility. The robot will learn about the patients’ regular activities. It
% will use this knowledge to perform support tasks for the care staff and serve as an early warning
% system when patients vary from regular behaviour (e.g. wandering the corridors at night, falling
% over). In our security scenario we will explore how a robot can act as a security guard, performing
% patrols to learn the typical spatial-temporal structures in a building and notifying a human guard
% of suspicious variations from these.

% \section{Limitations and Further Work}
% 
% Two basic statistical models: Spectral-FOPP and POPP  have been proposed and evaluated. The combination of these two is able to extract temporal dynamics in the aggregate level of human activities from unreliable sensors, along with the ability to exploit this understanding for better exploration by an autonomous mobile robot. However, the spectral-POPP model could still be improved in the following two ways:
% \begin{enumerate}
%     \item In Chapter \ref{chap:popp_independent}, The Gamma filter approximates a sum of Gamma distributions with a single Gamma distribution assuming that the sensor performs rather reliable. Instead of using a single Gamma distribution to approximate a sum of $m$ Gamma distributions, $n$ gamma distributions, where $n$ is much smaller than $m$, could be used to improve the accuracy of the approximation to the posterior. This would promise to be more accurate than a single gamma, but more efficient than a histogram filter. Thus, it might be faster than the switching filter.
% 
%     \item The spectral-Poisson model (Spectral-FOPP) in Chapter \ref{chap:spectral_poisson} is a statistical model which is able, and only able, to capture the periodic structure of count data. It indirectly assumes that there is an underlying pattern governing the evolution of the parameter $\lambda$ of a Poisson process. The spectral-Poisson might not be able to capture other non-periodic structures governing the parameter $\lambda$, such as trends. 
% 
%         A Gaussian process modulated Poisson process might provide a better model for different structures which govern $\lambda$ overtime. Work from \cite{lloyd2015variational} presents a fully variational Bayesian inference scheme for continuous Gaussian-process modulated Poisson process. It provides a good estimators and is fast in estimating $\lambda$ of a Poisson process. An extension to this statistical model which embeds both trends and periodicity in the model might provide a solution to the limitations of Spectral-Poisson while being fully Bayesian. 
% \end{enumerate}

% Listing all limitations which have been mentioned in previous sections.
% Listing all further work which can extend this thesis.

\section*{Acknowledgment}

The research leading to these results has received funding from the European Union Seventh Framework Programme (FP7/2007-2013) under grant agreement No 600623, STRANDS. Nick Hawes was supported by UK Research and Innovation and EPSRC through the Robotics and Artificial Intelligence for Nuclear (RAIN) research hub [EP/R026084/1].


% An example of a double column floating figure using two subfigures.
% (The subfig.sty package must be loaded for this to work.)
% The subfigure \label commands are set within each subfloat command,
% and the \label for the overall figure must come after \caption.
% \hfil is used as a separator to get equal spacing.
% Watch out that the combined width of all the subfigures on a 
% line do not exceed the text width or a line break will occur.
%
%\begin{figure*}[!t]
%\centering
%\subfloat[Case I]{\includegraphics[width=2.5in]{box}%
%\label{fig_first_case}}
%\hfil
%\subfloat[Case II]{\includegraphics[width=2.5in]{box}%
%\label{fig_second_case}}
%\caption{Simulation results for the network.}
%\label{fig_sim}
%\end{figure*}
%
% Note that often IEEE papers with subfigures do not employ subfigure
% captions (using the optional argument to \subfloat[]), but instead will
% reference/describe all of them (a), (b), etc., within the main caption.
% Be aware that for subfig.sty to generate the (a), (b), etc., subfigure
% labels, the optional argument to \subfloat must be present. If a
% subcaption is not desired, just leave its contents blank,
% e.g., \subfloat[].


% An example of a floating table. Note that, for IEEE style tables, the
% \caption command should come BEFORE the table and, given that table
% captions serve much like titles, are usually capitalized except for words
% such as a, an, and, as, at, but, by, for, in, nor, of, on, or, the, to
% and up, which are usually not capitalized unless they are the first or
% last word of the caption. Table text will default to \footnotesize as
% the IEEE normally uses this smaller font for tables.
% The \label must come after \caption as always.
%
%\begin{table}[!t]
%% increase table row spacing, adjust to taste
%\renewcommand{\arraystretch}{1.3}
% if using array.sty, it might be a good idea to tweak the value of
% \extrarowheight as needed to properly center the text within the cells
%\caption{An Example of a Table}
%\label{table_example}
%\centering
%% Some packages, such as MDW tools, offer better commands for making tables
%% than the plain LaTeX2e tabular which is used here.
%\begin{tabular}{|c||c|}
%\hline
%One & Two\\
%\hline
%Three & Four\\
%\hline
%\end{tabular}
%\end{table}

\ifCLASSOPTIONcaptionsoff
  \newpage
\fi

% trigger a \newpage just before the given reference
% number - used to balance the columns on the last page
% adjust value as needed - may need to be readjusted if
% the document is modified later
%\IEEEtriggeratref{8}
% The "triggered" command can be changed if desired:
%\IEEEtriggercmd{\enlargethispage{-5in}}

% REFERENCES
\bibliographystyle{IEEEtran}
\bibliography{biblio}

% biography section with photo
%\begin{IEEEbiography}[{\includegraphics[width=1in,height=1.25in,clip,keepaspectratio]{mshell}}]{Michael Shell}
% \begin{IEEEbiography}{Ferdian Jovan}
% Biography text here.
% \end{IEEEbiography}

% biography section without photo
\begin{IEEEbiographynophoto}{Ferdian Jovan}
Biography text here.
\end{IEEEbiographynophoto}

\begin{IEEEbiographynophoto}{Jeremy Wyatt}
Biography text here.
\end{IEEEbiographynophoto}

\begin{IEEEbiographynophoto}{Nick Hawes}
Biography text here.
\end{IEEEbiographynophoto}

% % insert where needed to balance the two columns on the last page with
% % biographies
% %\newpage
% 
% \begin{IEEEbiographynophoto}{Jane Doe}
% Biography text here.
% \end{IEEEbiographynophoto}

% You can push biographies down or up by placing
% a \vfill before or after them. The appropriate
% use of \vfill depends on what kind of text is
% on the last page and whether or not the columns
% are being equalized.

%\vfill

% Can be used to pull up biographies so that the bottom of the last one
% is flush with the other column.
%\enlargethispage{-5in}

\end{document}
