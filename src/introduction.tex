\section{Introduction}
\label{sec:introduction}

% Autonomous mobile robots that work in human-centred environments are seen as a promising future application area for robotics systems. These systems are expected to leave carefully controlled industrial environments and come to assist us with menial tasks in our homes and offices. Possible applications include, for example, domestic robot assistants which help people to live independently for longer \cite{5751968}, delivery robots in hospitals, stock-keeping robots in supermarkets and warehouses, and security robots in factories.
% 
% Having robots operate in human populated environments requires modelling human activities. This is because any tasks or plans the robots have must take into account human activities. Since human activities involve many complex interactions they can be modelled at many levels of detail. These can range from recognising simple activities over a few seconds, a minute, or an hour; predicting what a person is going to do next; to determining whether a group of people are performing an activity together. In any scenario where a robot learns about human activities, it must first observe them. Thus, the first thing the robot needs to do is to plan to go to where people are. This problem of finding and observing people is the basic motivation for this thesis.  
% 
% To be where people are, the robot must first know when and where it is likely to see people. It becomes, however, a challenging problem if one tries to predict exactly where a particular individual will be, so as to observe that individual. There are two reasons for this. First, each individual roaming in an human populated environment is hard to re-recognise. Second, individual persons often travel long distances and visit places robots can not follow. Hence, instead of predicting where an individual will be, it would be easier to predict where the robot is likely to see many people, without regard for exactly who it may observe. This formulation would allow the robot to observe a sufficient amount of human activity, so as to learn the specific kinds of activity models mentioned previously.
% 
% The problem of this thesis can be loosely formulated as how to predict where many people are most likely to be and to go and observe them. Specifically, it requires the robot to go to where the aggregate level of human activity is highest. In addition, this thesis chooses to tackle the problem for the case where the robot runs for an extended period of time such days, or even weeks as it builds its models.
% 
% To tackle the formulated problem, a mobile robot must know not only where people are, but also when they are in those locations. It also needs a model which predicts how many people the robot will be likely to see in a particular place at a particular time, since people tend to move around from one location to another. 
% 
% An important restriction on using a mobile robot is that it can only be in one place at one time, so its observations are spatially restricted. As the robot moves around, it will only see particular locations infrequently. Thus its data for those locations will be temporally sparse. This adds yet another requirement to the problem formulation, where the robot must know how uncertain it is about how many people might be in one place at a particular time. This requirement is necessary since the robot has a limit to its operational life. We would therefore like it to optimise the time it takes to build its models. This introduces an exploration-exploitation trade-off problem \cite{wyatt1998exploration, 1413255, AUDIBERT20091876}, i.e. should the robot visit a familiar place, where it will probably see two people, or go to a new place, where it might see many more but might see no-one?  Another important restriction in mobile robotics is that robot sensors are unreliable. Any solution must take into account the unreliability of sensors. We may also require that it do so for when multiple sensors are involved.
