\section{Introduction}
\label{sec:introduction}

In recent years, autonomous mobile robots have started to come to assist humans with simple daily tasks in our homes and offices \cite{hawes2016strands}. As the robot is expected to work in dynamic environments, any behaviour model for the robot to adapt with humans is typically learnt over time from observations \cite{coppola2016learning}. Where these observations are made using sensor data, both humans and all the currently robotic sensory systems have some level of unreliability. This means collected data-sets that make up the observations typically contain systematic errors that lead to bias in the statistical estimate produced by the sensors. In this paper, we address the unreliability problem of sensors counting events by formulating a \textit{partially observable Poisson process} (POPP). As we may require it do so for when multiple sensors are involved in which one sensor may (or may not) correlate to one another, we extend the POPP model with several variations. A standard fully observable Poisson process (FOPP) is introduced to distinct the POPP and its derivatives from it.

Several practical contributions are made. First, a set of inference methods for a Poisson process which take into account the unreliability of either single or multiple sensory systems used to count events is addressed. Two tractable approximations to the posterior of such Poisson processes, which are combined into a switching filter, are presented to cope with the absence of a conjugate density. One (the \textit{Gamma filter}) is fast, but prone to drift from the true posterior in certain circumstances. The second (the \textit{histogram filter}) is slower but avoids drifts. Variations of these processes are introduced. First is the correlated POPP (C-POPP) to deal with sensors that are correlated to one another. The second variation is the POPP-Beta which deals with the unreliability of the observation model. We also combine the C-POPP and the POPP-Beta in the POPP-Dirichlet model that is able to deal with correlated sensors and the unreliability of the observation model. We demonstrate the properties of the POPP, its variations, and its filters by numerical simulations and real-world datasets.

Finally, we show the benefit of the POPP model, and its variation on a robot exploration task performed by a mobile robot. The resulting posterior of the POPP and its derivative is used to drive exporation by a mobile robot for a series of two week deployment. Robot exploration rises from the fact that any mobile robot needs to observe human activities/behaviour in order to learn and adapt in its human-populated environment. As the robot has a limit to its operational life, one would therefore like it to optimise the time it takes to build its models. We contrast this with an exploration method based on the FOPP model. We show that the exploration method involving corrections to systematic errors doubles the number of human encounters the robot experiences. 

% We show variations of the exploration methods based on optimistic predictions from the resulting posteriors of the first contribution. 
% BELUM BERES INI
% The second contribution is the variation of exploration methods for a mobile robot based on optimistic predictions from the resulting posteriors of the first contribution. As any mobile robot in human-populated environment needs to learn human behaviour/activity, it must first explore where activities are likely to happened and observe them. One would like the robot to observe a sufficient amount of human activity, so as to learn the specific kinds of activity models. However, as a mobile robot can only be in one place at one time, its observations are spatially restricted. Moreover, the robot has a limit to its operational life. We would therefore like it to optimise the time it takes to build its models. This introduces an exploration-exploitation trade-off problem \cite{wyatt1998exploration, 1413255, AUDIBERT20091876}, i.e. should the robot visit a familiar place, where it will probably observe two activities, or go to a new place, where it might observe many more but might see nothing? 

% is expected to work around and/or with humans, modelling human activities becomes a necessity. In any scenario where a robot learns about human activities, it must first explore where activities are likely to happened and observe them. 
% 
% One would like the robot to observe a sufficient amount of human activity, so as to learn the specific kinds of activity models. However, as a mobile robot can only be in one place at one time, its observations are spatially restricted. Moreover, the robot has a limit to its operational life. We would therefore like it to optimise the time it takes to build its models. This introduces an exploration-exploitation trade-off problem \cite{wyatt1998exploration, 1413255, AUDIBERT20091876}, i.e. should the robot visit a familiar place, where it will probably observe two activities, or go to a new place, where it might observe many more but might see nothing?
% 
% Another important restriction in mobile robotics is that robot sensors are unreliable. Any solution must take into account the unreliability of sensors. We may also require that it do so for when multiple sensors are involved. This means that collected data which are used for learning typically contain systematic errors that lead to bias in the statistical estimates produced by the event detection processes.

% To allow the robot to observe a sufficient amount of human activity, so as to learn the specific kinds of activity models, putting an exploration to places with high number of activities becomes a mandatory action. There are several reasons behind this. First, a mobile robot can only be in one place at one time, so its observations are spatially restricted.   
% 
% The problem of this thesis can be loosely formulated as how to predict where many people are most likely to be and to go and observe them. Specifically, it requires the robot to go to where the aggregate level of human activity is highest. In addition, this thesis chooses to tackle the problem for the case where the robot runs for an extended period of time such days, or even weeks as it builds its models.
% A key challenge is for the robot to be able to recognise and react adequately to dynamic changes that happen in human-centered environments, especially when the changes are the result of human behaviour. The learning process for an autonomous mobile robot to adapt to its environment takes a big portion of its lifetime and it needs to deal with the huge volume of experiences accessible to it as it runs for longer periods of time. One advantage of all these is that these experiences contain potentially useful information that can help the robot to eventually adapt to its environment.

% Taken together, the challenges and benefits faced by an autonomous mobile robot motivate this thesis to create long-term understanding of temporal dynamics of human activities, along with the ability to exploit this understanding for a better adaptation of an autonomous mobile robot. The robot is expected to demonstrate its ability to predict future activities based on its statistical model. It is also expected to be able to detect anomalies as sequences of very low likelihood data.


