\begin{abstract}
    We present practical Bayesian inferences and exploration methods for count data collected by autonomous robots with unreliable sensors in human-populated environments. It addresses the problem of drawing incorrect inferences from unreliable count data which affects the effectiveness of robot exploration in maximizing its interaction with humans. Two contributions are presented in this paper: (1) A set of inference methods for a Poisson process which takes into account the unreliability of the robotic sensory systems used to count events is proposed, investigated, and empirically tested. The actual posterior distribution of such Poisson processes is estimated via two tractable approximations. Variations of these processes are presented, in which (i) sensors are uncorrelated, (ii) sensors are correlated, (iii) the unreliability of the observation model, when built from data, is accounted for. (2) Several exploration methods based on optimistic predictions from the resulting posteriors of contribution (1) are proposed, and empirically evaluated. They are assessed by the way they improve the number of human encounters the robot experiences. The results indicate that addressing the unreliable sensors improve human encounters by at least a factor of 2.   

    % The Poisson assumption is a popular choice when data arises in the form of counts. In many applications such as in mobile robotics, such counts are prone to a systematic error due to unreliable sensor algorithms. Limited works have been done on the Poisson model when a full observability to count data is in question. Practical Bayesian estimators for a partially observable Poisson process (POPP) is presented in this paper to address this common problem. Variations of these processes are presented, in which (i) sensors are uncorrelated, (ii) sensors are correlated, (iii) the unreliability of the observation model, when built from data, is accounted for. The actual posterior distributions are estimated via two tractable approximations, which we combined in a switching filter. The filter enables efficient and accurate estimation of the posterior. The resulting posterior is used to drive exploration of a mobile robot with unreliable sensors. A detailed empirical analysis showing the benefit of correcting miscounts in robot exploration is presented.  
\end{abstract}

\begin{IEEEkeywords}
Poisson processes, partial observability, misclassified counts, robot exploration.
\end{IEEEkeywords}
