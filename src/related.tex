\section{Related Work}
\label{sec:related}

There are some existing works which address statistical models where the observation data are not fully observable. In some literature, this effect is regarded as misclassified counts. Misclassification happens when there are false positive counts or false negative counts (or possibly both). False positive counts, which can also be called the overcount, are when the count includes events other than those of interest. Whereas false negative counts, which can also be called the undercount, are when some of the events of interest are missed or omitted. Early work on this only involved binomial and multinomial models \cite{Bross1954,Chen1979,Hochberg1977,Tenenbein1970,Viana1993}. The first technique which was recorded to handle misclassification was double sampling. It was first introduced by Tenenbein to correct for misclassification of binomial data and obtain a maximum likelihood estimate \cite{Tenenbein1970}. The double sampling approach utilizes two search techniques to retrieve relevant information: an expensive classification technique to obtain the true count along with the false positive count and false negative count from typically a small sample set, and a less-expensive classification technique only for error-prone counts on a larger sample set. The results of both counts are then combined to obtain estimators for the Poisson rate $\lambda$, and also for the misclassification parameters. The work of Tenenbein was then extended by Chen \cite{Chen1979} and Hochberg \cite{Hochberg1977} to correct misclassified counts in categorical and multinomial models to obtain maximum likelihood estimates. The double sampling technique was also extended to incorporate prior distributions in the binomial model and the posterior was obtained via Bayesian estimation \cite{Viana1993}. Bekele extended the work of \cite{Viana1993} by introducing a weighted prior scheme and allowing for several sources of information, including expert opinions \cite{bekele1998binomial}.

Different than the case of binomial and multinomial models, only a few studies are found working on the effect of partial observability of the data on the Poisson distribution. Many deal with the undercount (or under-reporting) problem as this is quite a common problem. Whittemore and Gong estimated cervical cancer rates by taking into account false negative data \cite{whittemore1991}. Winkelmann and Zimmermann introduced a combination of a Poisson regression model with a logit model for under-reporting, yielding the Poisson-Logistic (Pogit) model \cite{winkelmann1993poisson}. They applied this to model the number of days employees were absent from a workplace. Dvorzak and Wagner borrowed the Pogit model and incorporated a small set of validation data, which is assumed available, to provide information about the true counts \cite{dvorzak2016}. A Bayesian analysis of the Poisson-Logistic model was performed and Bayesian variable selection was incorporated to identify regressors with a non-zero effect and also to restrict parameters of the Poisson-Logistic model.

Fewer worked on the Poisson model in the case when the count data may either be undercounting or overcounting \cite{sposto1992, bratcher2002, bratcher2004, stamey2005}. Sposto et al. estimated both cancer and non-cancer death rates, assuming false negatives are possible on both sides of these counts \cite{sposto1992}. The approach used by them followed the frequentist framework. Bratcher and Stamey, in \cite{bratcher2002}, used a Bayesian method to estimate Poisson rates in the presence of both undercounts and overcounts borrowing the double sampling technique introduced in \cite{Tenenbein1970}. They then extended their work to a fully Bayesian method for interval prediction of the unobservable actual count in future samples, given a current double sample \cite{bratcher2004}. Stamey and Young \cite{stamey2005} managed to obtained closed-form expressions for maximum likelihood estimators of the false negative rate, the false positive rate, and the Poisson rate for the model proposed in \cite{bratcher2002}. The estimators are straightforward to calculate and to interpret in terms of evaluating the effectiveness of using unreliable counts.

Our work is similar to the work in \cite{bratcher2002} to accurately estimate the parameter of a single Poisson process. They utilised the double sampling technique to obtain the true count together with false positive count and false negative count. The Poisson rate is estimated via an MCMC approximation due to no-closed form and expensive calculation of the posterior distribution of $\lambda$. Our work differs from theirs in that they are interested in estimation of a single Poisson process where the estimation is based on two counters with one always being a perfect counter. Here, we consider multiple unreliable counters, which may have correlations to one another, applied to estimating the parameter of a single Poisson process. To tackle the non-trivial density of the posterior, three precise and tractable Bayesian estimators are proposed. 


% Our work is an extension to the work in \cite{jovan18a} in accurately estimating the parameter of a single Poisson process to improve the prediction accuracy of Spectral-FOPP. We present three further extensions of the POPP model and its application to exploit the temporal dynamics in the aggregate level of human activities for better robot exploration.
