\section{Related Work}
\label{sec:related}

There are several variations of the basic Poisson process which has been recently used to model regularities in time series data in order to identify the irregular ones. Ihler et al. \cite{Ihler2006adaptive, Ihler2007detecting} described a modified Markov-modulated Poisson processes for detecting unusual data points or segments in time-series. The Poisson processes are used as probabilistic models for counting regular patterns and behaviour whereas the Markov chain is used to track the occurrence of anomalous events. Hutchins in \cite{hutchins2007countdata} extended the work of \cite{Ihler2006adaptive} from single to multiple counters, and applied it to estimating the occupancy level of a building. 

The work of \cite{hutchins2007countdata} also took into consideration the effect of misclassification (under-or-over count) on the Poisson distribution. Only a few studies are found studying this effect. Bratcher and Stamey used a Bayesian method to estimate Poisson rates in the presence of both undercounts and overcounts. They used the double sampling technique where the first sample is searched with both a fallible and an infallible method and the second sample is searched with only a fallible method \cite{bratcher2002}. It was then extended to a fully Bayesian method for interval prediction of the unobservable actual count in a future sample, given a current double sample \cite{bratcher2004}. Stamey and Young \cite{stamey2005} managed to obtained closed-form expressions for maximum likelihood estimators (MLEs) of the false negative rate, the false positive rate, and the Poisson rate for the model proposed in \cite{bratcher2002}. The estimators are straightforward to calculate and to interpret in terms of evaluating the effectiveness of using unreliable counts.

% This work is closely related to  \cite{hutchins2007countdata}. They used multiple unreliable counters, each at a different exit or entrance. Thus, each sensor is associated with a different Poisson process. For each entrance or exit they used a MMNHPP to estimate the arrival rate and a noise model to capture under-and over-counting. Their interest is in estimating a single latent variable influencing the arrival rates at multiple exits or entrances. In our case, we are interested in multiple unreliable sensors estimating the parameter of a single Poisson process.  

Our work is an extension to the work in \cite{jovan18a} in accurately estimating the parameter of a single Poisson process to improve the prediction accuracy of Spectral-FOPP. We present three further extensions of the POPP model and its application to exploit the temporal dynamics in the aggregate level of human activities for better robot exploration.
